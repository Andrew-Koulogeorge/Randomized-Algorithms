\documentclass[12pt]{article}
\usepackage{lipsum}
\usepackage{ifthen}
\usepackage{tikz-cd}
\usepackage{soul}
\usepackage{fullpage, prettyref}
\usepackage{fancyhdr}
\usepackage{palatino}
\usepackage[T1]{fontenc}
\usepackage{ulem}
\usepackage{graphicx}
\usepackage{ifthen}
\usepackage{algorithm}

\usepackage{a4}
\usepackage[margin=1in]{geometry} 
\usepackage[noend]{algpseudocode}
\usepackage{amsmath, amssymb, amsthm, mathtools}
\usepackage{setspace}
\usepackage{multicol}
\usepackage{array}
\usepackage{enumitem}
\usepackage{csquotes}
\usepackage{url}
\usepackage{boxedminipage}
\usepackage{wrapfig}
\usepackage{color, xcolor}
\usepackage{transparent}
\usepackage{parskip}
\usepackage{varwidth}
\usepackage{bm}
\usepackage{xspace}
\usepackage{url}
\usepackage{fullpage, prettyref}
\usepackage{boxedminipage}
\usepackage{wrapfig}
\usepackage{ifthen}
\usepackage{framed}
\usepackage{mdframed}
\usepackage[colorlinks=true,urlcolor=blue,linkcolor=blue,citecolor=violet,pdfstartview=FitH]{hyperref}
\usepackage[noabbrev,nameinlink]{cleveref}
\usepackage{enumitem}
\usepackage{tikz}
\usepackage{mathrsfs} 
\usepackage{bbm}
\usepackage[backend=biber,bibencoding=ascii,style=alphabetic,sorting=none]{biblatex}

% config
\renewcommand{\headrulewidth}{0pt}
\renewcommand{\footrulewidth}{0pt}
\pagestyle{fancy}
\fancyhead{}
\fancyfoot{}
\fancyfoot[R]{\thepage}

% Algebra
\newcommand{\unit}[1]{#1^{\times}}
\DeclareMathOperator{\Orb}{Orb}
\DeclareMathOperator{\Stab}{Stab}
\DeclareMathOperator{\Hom}{Hom}
\DeclareMathOperator{\End}{End}
\DeclareMathOperator{\Aut}{Aut}
\DeclareMathOperator{\Gal}{Gal}
\DeclareMathOperator{\rank}{rk}
\DeclareMathOperator{\img}{img}
\DeclareMathOperator{\id}{id}
\DeclareMathOperator{\ch}{ch}
\DeclareMathOperator{\tr}{tr}
\DeclareMathOperator{\codim}{codim}
\DeclareMathOperator{\ann}{ann}
\DeclareMathOperator{\ev}{ev}
\DeclareMathOperator{\proj}{proj}
\DeclareMathOperator{\ang}{ang}
\DeclareMathOperator{\GL}{GL}
\DeclareMathOperator{\Ob}{Ob}
\newcommand{\category}[1]{\mathsf{\underline{#1}}}
\newcommand{\acts}{\circlearrowleft}
\newcommand{\tleq}{\unlhd}

% Boolean logic/algorithms
\newcommand{\cor}{\textsc{ or }}
\newcommand{\cand}{\textsc{ and }}
\newcommand{\xor}{\oplus}
\newcommand{\true}{\textsc{True}}
\newcommand{\false}{\textsc{False}}
\newcommand{\nil}{\textsc{Nil}}
\newcommand{\alg}[1]{\textsc{#1}}
\DeclareMathOperator{\cmod}{mod}
\newcommand{\eq}{\leftarrow}
\newcommand{\chapterref}[2]{[\S #1, #2]}

% math formatting
\def\bar{\overline}
\newcommand{\script}[1]{\mathcal{#1}}
\newcommand{\fancyscript}[1]{\mathscr{#1}}


% number sets
\newcommand{\N}{\mathbb{N}}
\newcommand{\Q}{\mathbb{Q}}
\newcommand{\R}{\mathbb{R}}
\newcommand{\Z}{\mathbb{Z}}
\newcommand{\C}{\mathbb{C}}
\newcommand{\F}{\mathbb{F}}

%Paired Delims
\def\set#1{\left\{ #1 \right\}}
\def\angle#1{\left\langle #1 \right\rangle}
\def\sqbrackets#1{\left[ #1 \right]}
\def\brackets#1{\left( #1 \right)}
\DeclarePairedDelimiter\ceil{\lceil}{\rceil}
\DeclarePairedDelimiter\floor{\lfloor}{\rfloor}
\newcommand{\abs}[1]{\left\lvert #1 \right\rvert}

%problem environment
\newenvironment{problem}[2][Problem]{\begin{trivlist}
\item[\hskip \labelsep {\bfseries #1}\hskip \labelsep {\bfseries #2.}]}{\end{trivlist}
\vspace{1em}
}

%problems list environment
\newlist{subprob}{enumerate}{1}
\setlist[subprob]{label={\bfseries\itshape\alph*.}}

%solution environment
\newenvironment{solution}[1][Solution]{
    \setcounter{nlem}{0}
    \vspace{-2ex}
  \begin{proof}[#1]
}{
  \phantom\qedhere\end{proof}
}


% colored boxes
\newenvironment{answerbox}{
      \begin{mdframed}[backgroundcolor=green!30, innertopmargin=12px, topline=false,rightline=false,leftline=false,bottomline=false]}{
  \end{mdframed}%
}

\newenvironment{thmbox}{
        \vskip2ex
          \begin{mdframed}[backgroundcolor=blue!10, topline=false,rightline=false,leftline=false,bottomline=false]}{
  \end{mdframed}
}

\newenvironment{algbox}[1]{
        \vskip2ex
            \begin{mdframed}[backgroundcolor=yellow!10, topline=false,rightline=false,leftline=false,bottomline=false, userdefinedwidth=.7\linewidth, align=center]
              \vspace{1ex}
              \hrulefill
              \vspace{-1.5ex}
  
              {\bf #1}
  
              \vspace{-2.5ex}
              \hrulefill
              }{
            \end{mdframed}
}

\newenvironment{algboxw}[2]{
        \vskip2ex
            \begin{mdframed}[backgroundcolor=yellow!10, topline=false,rightline=false,leftline=false,bottomline=false, userdefinedwidth=#2\linewidth, align=center]
              \vspace{1ex}
              \hrulefill
              \vspace{-1.5ex}
  
              {\bf #1}
  
              \vspace{-2.5ex}
              \hrulefill
              }{
            \end{mdframed}
}

\newenvironment{algboxnt}{
        \vskip2ex
            \begin{mdframed}[backgroundcolor=yellow!10, topline=false,rightline=false,leftline=false,bottomline=false, userdefinedwidth=.7\linewidth, align=center]\vspace{1.5ex}}{
            \end{mdframed}
}

% theorem envs
\theoremstyle{definition}
\newtheorem*{lem}{Lemma}
\newtheorem*{claim}{Claim}
\newtheorem*{thm}{Theorem}
\newtheorem*{defn}{Definition}
\newtheorem*{coro}{Corollary}
\newtheorem*{rmk}{Remark}
\newtheorem*{prop}{Proposition}
\newtheorem*{example}{Example}
\newtheorem*{probstatement}{Problem}
\newtheorem{nlem}{Lemma}
\newtheorem{ndefn}[nlem]{Definition}
\newtheorem{nthm}[nlem]{Theorem}
\newtheorem{ncor}[nlem]{Corollary}
\newtheorem{nprop}[nlem]{Proposition}
\newtheorem{nrmk}[nlem]{Remark}


% Probability
\newcommand{\Prob}[1]{\mathbb{P}\left[#1\right]}
\newcommand{\Exp}[1]{\mathbf{E}\left[#1\right]}
\newcommand{\Var}[1]{\mathbf{Var}\left[#1\right]}
\newcommand{\Cov}[1]{\mathbf{Cov}[#1]}
\renewcommand{\Pr}{\Prob}
\DeclareMathOperator{\Bin}{Bin}
\DeclareMathOperator{\Pois}{Pois}
\DeclareMathOperator{\Geom}{Geom}
\DeclareMathOperator{\Normal}{\mathcal{N}}


% Misc commands
\newcommand{\sgn}{\mathrm{sgn}}
\newcommand{\poly}{\mathrm{poly}}
\newcommand{\polylog}{\mathrm{polylog}}
\newcommand{\contradiction}{\hfill$\Rightarrow\!\Leftarrow$}
\newcommand{\writer}{Andrew Koulogeorge, Eric Richardson}
\newcommand{\email}{andrew.j.koulogeorge.24@dartmouth.edu}
\newcommand{\TODO}[1]{\textcolor{red}{\bf [TODO: #1] }}
\newcommand{\note}[1]{\textcolor{blue}{\bf [Note: #1] }}
\newcommand{\alert}[1]{\textcolor{red}{\bf [#1]}}
\renewcommand{\epsilon}{\varepsilon}
\renewcommand{\phi}{\varphi}
\newcommand{\lnorm}[2]{\left\lVert#1\right\rVert_{#2}}
\DeclareMathOperator*{\argmin}{argmin}
\DeclareMathOperator*{\argmax}{argmax}
% PSet Titles
\newcommand{\normaltitle}[3]{
    \begin{center}
        {\LARGE #2}
        \vskip-1ex
        {\large \writer}
        \vskip-1ex
        {\large Date: \today}
        \vskip-1ex
        {\large Course: #1, ~#3}
        \vskip0.1ex
        \vspace{2ex}
        \hrule height 2pt
        \vspace{2ex}
    \end{center}
}

\newcommand{\fancytitle}[3]{
	{
            \rule{\textwidth}{1pt}
		\begin{center}
			{
                    \setlength{\parskip}{0mm} \setlength{\parindent}{0mm}
				{\textbf{#2} \hfill \writer}\\
				#1:~#3 \hfill Date: \today
                }
		\end{center}
            \vskip-2ex
            \rule{\textwidth}{1pt}
	}
 }
\everymath{\displaystyle}

\begin{document}

\fancytitle{COSC 34}{Problem Set 5}{Randomized Algorithms}

Collaboration Statement: Andrew Koulogeorge and Eric Richardson 


\begin{problem}{4: Pairwise Independent Hash Family}
\end{problem}
\begin{solution}
Let $D \subset U$ be a subset of $n$ distinct elements of $U$ and consider a hash table $T[1:n]$ of lists where we put $e \in D$ in the list $T[h(e)]$ for $j\in [n]$ and $e\in D$, let $X_{e,j}=1$ be the indicator variable for the event $h(e) = j$
\begin{enumerate}[label=(\alph*)]
\item We will apply the law of total probability. Fix an $\hat{e} \in U$ and define the collection of events $\Omega$ such that $\omega_i =\{h(\hat{e}) = i\}$. Note that $\Omega$ is mutually exclusive and collectively exhaustive. Thus, we can apply the law of total probability on the event that $h(e) = j$:
\[
\Pr{h(e) = j} = \sum_{i=1}^{n}{\Pr{h(e) = j \cap h(\hat{e}) = i }} = \sum_{i=1}^{n}{\frac{1}{n^2}} = \frac{1}{n} \qed
\]
\item Fix $e, \hat{e} \in U$. To show that $X_{e,j}$ and $X_{\hat{e},j}$ we need to show that:
\[
\forall~a,b~ \Pr{X_{e,j}=a \cap X_{\hat{e},j}=b} = \Pr{X_{e,j}=a} \Pr{X_{\hat{e},j}=b}
\]
Note that $X_{e,j}$ and $X_{\hat{e},j}$ only take on values $0$ and $1$ so we need to show this holds for each of the $4$ cases. Let $A = X_{e,j}=1$, $B = X_{e,j}=1$. We know from the previous part that $\Pr{A} = \Pr{B} = \frac{1}{n} \implies \Pr{\neg A} = \Pr{\neg B} = \frac{n-1}{n}$\\

$(1)$ From the definition of a pairwise independent hash family, we know that:
\[
\Pr{A \cap B} = \frac{1}{n^2} =\Pr{A}\Pr{B}
\]

$(2)$ From inclusion exclusion, we know that:
\[
\Pr{\neg A \cup \neg B} = \Pr{\neg A} + \Pr{\neg B} -\Pr{\neg A \cap \neg B}
\]
\[
\Pr{\neg A \cap \neg B} = \Pr{\neg A} + \Pr{\neg B} - \Pr{\neg A \cup \neg B}
\]
\[
\Pr{\neg A \cap \neg B} = \frac{2n(n-1)}{n^2}- \frac{n^2-1}{n^2} = \frac{n^2 - 2n + 1}{n^2} = \frac{(n-1)^2}{n^2}
\]
\[
\implies \Pr{\neg A \cap \neg B} = \frac{(n-1)^2}{n^2} = \Pr{\neg A}\Pr{\neg B} 
\]

$(3)$ Observe that $\Pr{\neg A \cap B} = \Pr{A \cap \neg B}$ since the random variables are identically distributed. We can use the fact that the sum of these $4$ events must equal $1$ to solve for $X = \Pr{\neg A \cap B} = \Pr{A \cap \neg B}$:
\[
\Pr{A \cap B} + \Pr{\neg A \cap \neg B} + 2X = 1
\]
\[
\frac{1}{n^2}+ \frac{(n-1)^2}{n^2} + 2X = 1
\]
\[
X = \frac{n-1}{n^2} = \Pr{\neg A}\Pr{B} = \Pr{A}\Pr{\neg B} \qed
\]
\item Just like the balls and bins case, we can express the load of a bin as a sum of indicators for each of the elements in $U$ which hash to the $j$th bin in $T$
\[
Z_j = \sum_{e\in U}{X_{e,j}}
\]
\[
\Exp{Z_j} = \sum_{e\in U}{\Exp{X_{e,j}}} = \sum_{e\in U}{\Pr{h(e) = j}} = 1
\]
where the last equality follows since there are $n$ elements and the probability any given element hashes to $j$ is $\frac{1}{n}$
\[
\Var{Z_j} = \Exp{Z^2_{j}} - \Exp{Z_{j}}^2
\]
\[
Z^2_{j} = \big(\sum_{e\in U}{X_{e,j}}\big)^2 = \sum_{i=1}^{n}{X^2_i} + \sum_{i\neq j}{X_iX_j}  
\]
\[
\Exp{Z^2_{j}} = \sum_{i=1}^{n}{\Exp{X^2_i}} + \sum_{i\neq j}{\Exp{X_i}\Exp{X_j}} = 1 + \frac{n(n-1)}{n^2} =2 -\frac{1}{n}
\]
The first expression is the definition of variance. The second is expanding out the square of the random load $Z_j$ into the squares and the cross terms. The third is applying linearity of expectation and noting that each of the cross term random variables are independent so their expectations factor. The final equality comes from the fact the the square of a Bernoulli random variable is itself and from the fact that that there a $n(n-1)$ cross terms in the second summation where each term is $\frac{1}{n^2}$. 
\[
\implies \Var{Z_j} = \Exp{Z^2_{j}} - \Exp{Z_{j}}^2 = 2 -\frac{1}{n} - 1 = 1 - \frac{1}{n}
\]
\item Let $Z = \max_{j}{Z_j}$ where $Z_j$ is the load on $T[j]$ after hashing all $n$ element from $D$ using a random $h\in H$. We want to show that:
\[
\Pr{Z \leq \Theta_n} = \Pr{\bigcap_{i=1}^n{Z_i \leq \Theta_n}} \geq \frac{2}{3}
\]
We will do this by upper bounding $\Pr{Z > \Theta_n}$:
\[
\Pr{Z > \Theta_n} = \Pr{\bigcup_{i=1}^n{Z_i > \Theta_n}} \leq \sum_{i=1}^n{\Pr{Z_i > \Theta_n}} = n\Pr{Z_j > \Theta_n}
\]
Where the second to last inequality follows from union bound and the final equality follows since each of the load random variables are identically distributed, where $j$ can be the load of any of the slots. Now we want to use the fact from part b that $Z_j$ can be expressed as the sum of pairwise independent indicators and we can apply a theorem from week $4$ which gives us a Chebyshev like bound:
\[
Z_j = \sum_{e\in U}{X_{e,j}} \implies \Exp{Z_j} = 1
\]
\[
\Pr{|Z_j -1| \geq c} \leq \frac{1}{c^2}
\]
Note that since $Z_j$ only takes on integer values larger than $0$, if $c >1$ then we can drop the absolute value signs:
\[
\Pr{Z_j \geq c + 1} \leq \frac{1}{c^2}
\]
Note that $Z_j$ only takes on positive integer values so the following equality holds: 
\[
\Pr{Z_j \geq c + 1} = \Pr{Z_j > c}
\]
\[
\implies \Pr{Z_j > c} \leq \frac{1}{c^2}
\]
Let $c = \Theta_n$
\[
\Pr{Z > \Theta_n} = n\Pr{Z_j > \Theta_n} \leq \frac{n}{\Theta^2_n} \leq \frac{1}{3}
\]
\[
\Pr{Z > \Theta_n} \leq \frac{1}{3} \implies \Pr{Z \leq \Theta_n} \geq \frac{2}{3}
\]
where the last inequality holds if $\Theta_n \geq \sqrt{3n}$.\qed


\end{enumerate}
\end{solution}

\end{document}