\documentclass[12pt]{article}
\usepackage{mathtools}
\DeclarePairedDelimiter\ceil{\lceil}{\rceil}
\DeclarePairedDelimiter\floor{\lfloor}{\rfloor}
\usepackage{lipsum}

\usepackage{ifthen}
\usepackage{tikz-cd}
\usepackage{soul}
\usepackage{fullpage, prettyref}
\usepackage{fancyhdr}
\usepackage{palatino}
\usepackage[T1]{fontenc}
\usepackage{ulem}
\usepackage{graphicx}
\usepackage{ifthen}
\usepackage{algorithm}

\usepackage{a4}
\usepackage[margin=1in]{geometry} 
\usepackage[noend]{algpseudocode}
\usepackage{amsmath, amssymb, amsthm, mathtools}
\usepackage{setspace}
\usepackage{multicol}
\usepackage{array}
\usepackage{enumitem}
\usepackage{csquotes}
\usepackage{url}
\usepackage{boxedminipage}
\usepackage{wrapfig}
\usepackage{color, xcolor}
\usepackage{transparent}
\usepackage{parskip}
\usepackage{varwidth}
\usepackage{bm}
\usepackage{xspace}
\usepackage{url}
\usepackage{fullpage, prettyref}
\usepackage{boxedminipage}
\usepackage{wrapfig}
\usepackage{ifthen}
\usepackage{framed}
\usepackage{mdframed}
\usepackage[colorlinks=true,urlcolor=blue,linkcolor=blue,citecolor=violet,pdfstartview=FitH]{hyperref}
\usepackage[noabbrev,nameinlink]{cleveref}
\usepackage{enumitem}
\usepackage{tikz}
\usepackage{mathrsfs} 
\usepackage{bbm}
\usepackage[backend=biber,bibencoding=ascii,style=alphabetic,sorting=none]{biblatex}

% config
\renewcommand{\headrulewidth}{0pt}
\renewcommand{\footrulewidth}{0pt}
\pagestyle{fancy}
\fancyhead{}
\fancyfoot{}
\fancyfoot[R]{\thepage}

% Algebra
\newcommand{\unit}[1]{#1^{\times}}
\DeclareMathOperator{\Orb}{Orb}
\DeclareMathOperator{\Stab}{Stab}
\DeclareMathOperator{\Hom}{Hom}
\DeclareMathOperator{\End}{End}
\DeclareMathOperator{\Aut}{Aut}
\DeclareMathOperator{\Gal}{Gal}
\DeclareMathOperator{\rank}{rk}
\DeclareMathOperator{\img}{img}
\DeclareMathOperator{\id}{id}
\DeclareMathOperator{\ch}{ch}
\DeclareMathOperator{\tr}{tr}
\DeclareMathOperator{\codim}{codim}
\DeclareMathOperator{\ann}{ann}
\DeclareMathOperator{\ev}{ev}
\DeclareMathOperator{\proj}{proj}
\DeclareMathOperator{\ang}{ang}
\DeclareMathOperator{\GL}{GL}
\DeclareMathOperator{\Ob}{Ob}
\newcommand{\category}[1]{\mathsf{\underline{#1}}}
\newcommand{\acts}{\circlearrowleft}
\newcommand{\tleq}{\unlhd}

% Boolean logic/algorithms
\newcommand{\cor}{\textsc{ or }}
\newcommand{\cand}{\textsc{ and }}
\newcommand{\xor}{\oplus}
\newcommand{\true}{\textsc{True}}
\newcommand{\false}{\textsc{False}}
\newcommand{\nil}{\textsc{Nil}}
\newcommand{\alg}[1]{\textsc{#1}}
\DeclareMathOperator{\cmod}{mod}
\newcommand{\eq}{\leftarrow}
\newcommand{\chapterref}[2]{[\S #1, #2]}

% math formatting
\def\bar{\overline}
\newcommand{\script}[1]{\mathcal{#1}}
\newcommand{\fancyscript}[1]{\mathscr{#1}}


% number sets
\newcommand{\N}{\mathbb{N}}
\newcommand{\Q}{\mathbb{Q}}
\newcommand{\R}{\mathbb{R}}
\newcommand{\Z}{\mathbb{Z}}
\newcommand{\C}{\mathbb{C}}
\newcommand{\F}{\mathbb{F}}

%Paired Delims
\def\set#1{\left\{ #1 \right\}}
\def\angle#1{\left\langle #1 \right\rangle}
\def\sqbrackets#1{\left[ #1 \right]}
\def\brackets#1{\left( #1 \right)}
\DeclarePairedDelimiter\ceil{\lceil}{\rceil}
\DeclarePairedDelimiter\floor{\lfloor}{\rfloor}
\newcommand{\abs}[1]{\left\lvert #1 \right\rvert}

%problem environment
\newenvironment{problem}[2][Problem]{\begin{trivlist}
\item[\hskip \labelsep {\bfseries #1}\hskip \labelsep {\bfseries #2.}]}{\end{trivlist}
\vspace{1em}
}

%problems list environment
\newlist{subprob}{enumerate}{1}
\setlist[subprob]{label={\bfseries\itshape\alph*.}}

%solution environment
\newenvironment{solution}[1][Solution]{
    \setcounter{nlem}{0}
    \vspace{-2ex}
  \begin{proof}[#1]
}{
  \phantom\qedhere\end{proof}
}


% colored boxes
\newenvironment{answerbox}{
      \begin{mdframed}[backgroundcolor=green!30, innertopmargin=12px, topline=false,rightline=false,leftline=false,bottomline=false]}{
  \end{mdframed}%
}

\newenvironment{thmbox}{
        \vskip2ex
          \begin{mdframed}[backgroundcolor=blue!10, topline=false,rightline=false,leftline=false,bottomline=false]}{
  \end{mdframed}
}

\newenvironment{algbox}[1]{
        \vskip2ex
            \begin{mdframed}[backgroundcolor=yellow!10, topline=false,rightline=false,leftline=false,bottomline=false, userdefinedwidth=.7\linewidth, align=center]
              \vspace{1ex}
              \hrulefill
              \vspace{-1.5ex}
  
              {\bf #1}
  
              \vspace{-2.5ex}
              \hrulefill
              }{
            \end{mdframed}
}

\newenvironment{algboxw}[2]{
        \vskip2ex
            \begin{mdframed}[backgroundcolor=yellow!10, topline=false,rightline=false,leftline=false,bottomline=false, userdefinedwidth=#2\linewidth, align=center]
              \vspace{1ex}
              \hrulefill
              \vspace{-1.5ex}
  
              {\bf #1}
  
              \vspace{-2.5ex}
              \hrulefill
              }{
            \end{mdframed}
}

\newenvironment{algboxnt}{
        \vskip2ex
            \begin{mdframed}[backgroundcolor=yellow!10, topline=false,rightline=false,leftline=false,bottomline=false, userdefinedwidth=.7\linewidth, align=center]\vspace{1.5ex}}{
            \end{mdframed}
}

% theorem envs
\theoremstyle{definition}
\newtheorem*{lem}{Lemma}
\newtheorem*{claim}{Claim}
\newtheorem*{thm}{Theorem}
\newtheorem*{defn}{Definition}
\newtheorem*{coro}{Corollary}
\newtheorem*{rmk}{Remark}
\newtheorem*{prop}{Proposition}
\newtheorem*{example}{Example}
\newtheorem*{probstatement}{Problem}
\newtheorem{nlem}{Lemma}
\newtheorem{ndefn}[nlem]{Definition}
\newtheorem{nthm}[nlem]{Theorem}
\newtheorem{ncor}[nlem]{Corollary}
\newtheorem{nprop}[nlem]{Proposition}
\newtheorem{nrmk}[nlem]{Remark}


% Probability
\newcommand{\Prob}[1]{\mathbb{P}\left[#1\right]}
\newcommand{\Exp}[1]{\mathbf{E}\left[#1\right]}
\newcommand{\Var}[1]{\mathbf{Var}\left[#1\right]}
\newcommand{\Cov}[1]{\mathbf{Cov}[#1]}
\renewcommand{\Pr}{\Prob}
\DeclareMathOperator{\Bin}{Bin}
\DeclareMathOperator{\Pois}{Pois}
\DeclareMathOperator{\Geom}{Geom}
\DeclareMathOperator{\Normal}{\mathcal{N}}


% Misc commands
\newcommand{\sgn}{\mathrm{sgn}}
\newcommand{\poly}{\mathrm{poly}}
\newcommand{\polylog}{\mathrm{polylog}}
\newcommand{\contradiction}{\hfill$\Rightarrow\!\Leftarrow$}
\newcommand{\writer}{Andrew Koulogeorge, Eric Richardson}
\newcommand{\email}{andrew.j.koulogeorge.24@dartmouth.edu}
\newcommand{\TODO}[1]{\textcolor{red}{\bf [TODO: #1] }}
\newcommand{\note}[1]{\textcolor{blue}{\bf [Note: #1] }}
\newcommand{\alert}[1]{\textcolor{red}{\bf [#1]}}
\renewcommand{\epsilon}{\varepsilon}
\renewcommand{\phi}{\varphi}
\newcommand{\lnorm}[2]{\left\lVert#1\right\rVert_{#2}}
\DeclareMathOperator*{\argmin}{argmin}
\DeclareMathOperator*{\argmax}{argmax}
% PSet Titles
\newcommand{\normaltitle}[3]{
    \begin{center}
        {\LARGE #2}
        \vskip-1ex
        {\large \writer}
        \vskip-1ex
        {\large Date: \today}
        \vskip-1ex
        {\large Course: #1, ~#3}
        \vskip0.1ex
        \vspace{2ex}
        \hrule height 2pt
        \vspace{2ex}
    \end{center}
}

\newcommand{\fancytitle}[3]{
	{
            \rule{\textwidth}{1pt}
		\begin{center}
			{
                    \setlength{\parskip}{0mm} \setlength{\parindent}{0mm}
				{\textbf{#2} \hfill \writer}\\
				#1:~#3 \hfill Date: \today
                }
		\end{center}
            \vskip-2ex
            \rule{\textwidth}{1pt}
	}
 }
\everymath{\displaystyle}

\begin{document}

\fancytitle{COSC 34}{Problem Set 7}{Randomized Algorithms}

Collaboration Statement: Andrew Koulogeorge and Eric Richardson 

\begin{problem}{3: Single Element Recovery} 
\end{problem}

\begin{solution} \ \\
Given a binary vector $x \in \{0,1\}^n$, we wish to find an index $i\in [n]$ such that $x_i=1$ for some $x_i\in x$. A sum query is the sum of a subset of elements from $x$ denoted by $x(A_k) = \sum_{j\in A_k}{x_j}$ where $A_k$ is a subset of index's from $[n]$.

\begin{enumerate}[label=(\alph*)]
\item Suppose that we knew that exactly $1$ entry $x_i \in x$ was $1$ and the rest were zero $\implies \sum_{i=1}^{n}{x_i} = 1$. We wish to identify a deterministic algorithm such that $\forall~x\in \{0,1\}^n ~s.t~\sum_{i=1}^{n}{x_i} = 1$, our algorithm computes at most $\log_2(n)$ sum queries and returns the index $i\in [n]~s.t~x_i=1$.\\

We will compute which index $i\in [n]$ is $1$ in $x$ by constructing the binary representation of $i$ using each value of $x(A_k)$ as a single bit. Consider the binary string formed by the concatenation of all of the results of $x(A_k)$ in decreasing order:
\[
(x(A_{\log_{2}{n}}), x(A_{\log_2{n}-1}), ..., x(A_{2}),x(A_{1}))
\]
We wish to construct these subsets such that when each query sum is computed the above tuple equals $i-1$ in binary. For the construction, consider each index $i=1\rightarrow [n]$. For each $i$, we compute the binary representation of $i-1$. Note that this binary representation has at most $\log_{2}(n)$ bits. For each subset $A_k$, add $i$ to the subset $A_k$ if the $kth$ digit in the binary representation of $i-1$ is $1$. Based on this construction, when the $ith$ element in $x$ is set to $1$ the tuple of query sums above will spell out the index $i$ in binary minus $1$ (in the case where $n$ is a power of $2$, all $1s  \implies i=n$ and all $0s \implies i=1$. \qed

\item Now suppose that $\sum_{i=1}^{n}{x_i} = d$. We still wish to recover an any $i\in [n]$ where $x_i=1$. Define a "nonzero index" as an index $i$ such that $x_i=1$. Also recall that we need to ask all of our questions about query sums at once. Consider taking $T$ subsets from $x$: $R_1, R_2, ..., R_T$ where   each element of $x$ is sampled into $R_k$ independently with probability $\frac{1}{d}$. Note that each of these subsets are i.i.d. 

\textbf{Procedure:} For each $R_k$, we can construct a bit vector $x_k$ which contains the elements of $x$ indexed by the elements of $R_k$. We can then apply the algorithm from part (a) on this collection of bit vectors AND compute the sum of all the entries of $x_k$. This requires at most $\log_2(n)T + T$ sum queries since the length of $x_t$ is at most $n$. First, we check if one of these subsets has exactly $1$ nonzero index by using the sum query over all the index's of the subset. If we sampled a subset $R$ with this property, we will be able to  deterministically find a nonzero index by applying the algorithm from part (a). Thus, all that is left to show is that the probability that all of the subsets do not contain exactly one nonzero index is small. Let $Q_{i}$ be the number of nonzero index's in the subset $R_i$:
\[
\Pr{Fail} = \Pr{Q_{1} \neq 1 \cap Q_{2} \neq 1 \cap ... \cap Q_{T} \neq 1} = \prod_{i = 1}^{T}{\Pr{Q_i \neq 1}}
\]
Since $Q_i$ is a binomial random variable with $d$ trails, one for each of the nonzero index's, each having a success probability of $\frac{1}{d}$ we have:
\[
\Pr{Q_i = 1} = \binom{d}{1}\frac{1}{d}(1 - \frac{1}{d})^{d-1} \geq 1 - \frac{d-1}{d}= \frac{1}{d}
\]
\[
\implies \Pr{Q_i \neq 1} \leq 1 -\frac{1}{d}
\]
where the equality is the definition of the binomial pmf and the inequality follows from $(1 - x)^y \geq 1-xy$ for $x \geq 0$ .
Since each $Q^i$ is identically distributed we can apply this upper bound for each of the $T$ terms:
\[
\Pr{Fail} = \prod_{i = 1}^{T}{\Pr{Q_i \neq 1}} \leq (1 - \frac{1}{d})^T \leq \exp{\frac{-T}{d}} \leq \delta
\]
\[
\implies T \geq d\log{\frac{1}{\delta}}
\]

Thus, by using $d\log{\frac{1}{\delta}} (\log_{2}{n} + 1) = O(\log{\frac{1}{\delta}}\log_{2}{n})$ query sums we can find a nonzero index of $x$ with probability at least $1-\delta$. \qed
\item Now suppose that instead of knowing the exact value of $\sum_{i=1}^{n}{x_i}$ beforehand, we know that:
\[
d \leq \sum_{i=1}^{n}{x_i} \leq 2d
\]
Let $y=\sum_{i=1}^{n}{x_i}$. I claim that we can still apply the algorithm in the previous part except now we will sample each element from $x$ into $R_k$ with probability $\frac{1}{2d}$. The difference is that now the Binomial random variable $Q_i$ has $y$ trials and has a probability of success of $\frac{1}{2d}$ :
\[
d \leq y \leq 2d \implies \frac{1}{2} \leq \frac{y}{2d} \leq 1
\]
\[
\Pr{Q_i = 1} = \binom{y}{1}\frac{1}{2d}(1 - \frac{1}{2d})^{y-1} \geq \frac{y}{2d}(1-\frac{y-1}{2d}) \geq \frac{1}{2}(1-\frac{y-1}{2d}) = \frac{1}{2}(1-\frac{y}{2d} + \frac{1}{2d}) \geq \frac{1}{4d}
\]

\[
\implies \Pr{Q_i \neq 1} \leq 1 -\frac{1}{4d}
\]
The same analysis from the previous question now holds. We get a larger constant value on $d$ but it still requires $O(\log{\frac{1}{\delta}}\log_{2}{n})$ query sums to find a nonzero index with probability $\geq 1-\delta$. Qualitatively, if we can construct an estimate for $y$ within an interval $d \leq y \leq 2d$ we can find a non negative index in $x$ w.h.p \qed

\item Suppose that $y = \sum_{i=1}^n{x_i}$ is completely unknown. Let $d=2^i$ for $1 \leq i \leq \floor*{\log_{2}{n}}$. If $2 \leq y \leq n \implies \exists~k, ~1\leq k \leq \floor*{\log_{2}{n}} ~s.t~ ~ 2^k \leq y < 2^{k+1} \implies d \leq y < 2d$. In short, we know there exists an interval around $y$ which is length $d$, but since we do not know $d$ up front, we do not know the exact value of $d$ such that sampling $iid$ from $[n]$ with probability $\frac{1}{2d}$ inherits the results from part (c). If we knew $d$, we could apply part (c) and recover a nonzero index in $O(\log{\frac{1}{\delta}}\log_{2}{n})$ query sums.

Here is the idea to "get" $d$: sample  $\floor*{\log_{2}{n}}$ index subsets $R_1, R_2, ... R_{\floor*{\log_{2}{n}}}$ from $[n]$ where $R_i$ samples each index from $[n]$ independently with probability $\frac{1}{2^{i+1}}$. We dont know for which $k$ $y$ will lie in an interval $2^k \leq y < 2^{k+1}$, so exhaust all possible upper bounds by sampling $\floor*{\log_{2}{n}}$ different subsets $\implies $ there will exist a subset $R_k$ for which we can apply the results from part $c$. 

The boosting is being applied on each of the subsets individually. For each $R_1, R_2, ... R_{\floor*{\log_{2}{n}}}$ apply the algorithm from part $b$ which takes a total of $O(log^{2}{n})$ queries. We showed that one of these subsets will be sampling with a probability within a constant factor of $\frac{1}{y}$ so we can boost this process $O(log{\frac{1}{\delta}})$ times $\implies $ if we dont know the number of nonzero entries in $x$ upfront, we can still recover a nonzero index $i$ with $O(log^{2}{n}log{\frac{1}{\delta}}))$ queries with probability at least $1-\delta$ \qed

\end{enumerate}
\end{solution}


\end{document}