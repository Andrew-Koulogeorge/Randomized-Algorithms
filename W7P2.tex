\documentclass[12pt]{article}
\usepackage{mathtools}
\DeclarePairedDelimiter\ceil{\lceil}{\rceil}
\DeclarePairedDelimiter\floor{\lfloor}{\rfloor}
\usepackage{lipsum}

\usepackage{ifthen}
\usepackage{tikz-cd}
\usepackage{soul}
\usepackage{fullpage, prettyref}
\usepackage{fancyhdr}
\usepackage{palatino}
\usepackage[T1]{fontenc}
\usepackage{ulem}
\usepackage{graphicx}
\usepackage{ifthen}
\usepackage{algorithm}

\usepackage{a4}
\usepackage[margin=1in]{geometry} 
\usepackage[noend]{algpseudocode}
\usepackage{amsmath, amssymb, amsthm, mathtools}
\usepackage{setspace}
\usepackage{multicol}
\usepackage{array}
\usepackage{enumitem}
\usepackage{csquotes}
\usepackage{url}
\usepackage{boxedminipage}
\usepackage{wrapfig}
\usepackage{color, xcolor}
\usepackage{transparent}
\usepackage{parskip}
\usepackage{varwidth}
\usepackage{bm}
\usepackage{xspace}
\usepackage{url}
\usepackage{fullpage, prettyref}
\usepackage{boxedminipage}
\usepackage{wrapfig}
\usepackage{ifthen}
\usepackage{framed}
\usepackage{mdframed}
\usepackage[colorlinks=true,urlcolor=blue,linkcolor=blue,citecolor=violet,pdfstartview=FitH]{hyperref}
\usepackage[noabbrev,nameinlink]{cleveref}
\usepackage{enumitem}
\usepackage{tikz}
\usepackage{mathrsfs} 
\usepackage{bbm}
\usepackage[backend=biber,bibencoding=ascii,style=alphabetic,sorting=none]{biblatex}

% config
\renewcommand{\headrulewidth}{0pt}
\renewcommand{\footrulewidth}{0pt}
\pagestyle{fancy}
\fancyhead{}
\fancyfoot{}
\fancyfoot[R]{\thepage}

% Algebra
\newcommand{\unit}[1]{#1^{\times}}
\DeclareMathOperator{\Orb}{Orb}
\DeclareMathOperator{\Stab}{Stab}
\DeclareMathOperator{\Hom}{Hom}
\DeclareMathOperator{\End}{End}
\DeclareMathOperator{\Aut}{Aut}
\DeclareMathOperator{\Gal}{Gal}
\DeclareMathOperator{\rank}{rk}
\DeclareMathOperator{\img}{img}
\DeclareMathOperator{\id}{id}
\DeclareMathOperator{\ch}{ch}
\DeclareMathOperator{\tr}{tr}
\DeclareMathOperator{\codim}{codim}
\DeclareMathOperator{\ann}{ann}
\DeclareMathOperator{\ev}{ev}
\DeclareMathOperator{\proj}{proj}
\DeclareMathOperator{\ang}{ang}
\DeclareMathOperator{\GL}{GL}
\DeclareMathOperator{\Ob}{Ob}
\newcommand{\category}[1]{\mathsf{\underline{#1}}}
\newcommand{\acts}{\circlearrowleft}
\newcommand{\tleq}{\unlhd}

% Boolean logic/algorithms
\newcommand{\cor}{\textsc{ or }}
\newcommand{\cand}{\textsc{ and }}
\newcommand{\xor}{\oplus}
\newcommand{\true}{\textsc{True}}
\newcommand{\false}{\textsc{False}}
\newcommand{\nil}{\textsc{Nil}}
\newcommand{\alg}[1]{\textsc{#1}}
\DeclareMathOperator{\cmod}{mod}
\newcommand{\eq}{\leftarrow}
\newcommand{\chapterref}[2]{[\S #1, #2]}

% math formatting
\def\bar{\overline}
\newcommand{\script}[1]{\mathcal{#1}}
\newcommand{\fancyscript}[1]{\mathscr{#1}}


% number sets
\newcommand{\N}{\mathbb{N}}
\newcommand{\Q}{\mathbb{Q}}
\newcommand{\R}{\mathbb{R}}
\newcommand{\Z}{\mathbb{Z}}
\newcommand{\C}{\mathbb{C}}
\newcommand{\F}{\mathbb{F}}

%Paired Delims
\def\set#1{\left\{ #1 \right\}}
\def\angle#1{\left\langle #1 \right\rangle}
\def\sqbrackets#1{\left[ #1 \right]}
\def\brackets#1{\left( #1 \right)}
\DeclarePairedDelimiter\ceil{\lceil}{\rceil}
\DeclarePairedDelimiter\floor{\lfloor}{\rfloor}
\newcommand{\abs}[1]{\left\lvert #1 \right\rvert}

%problem environment
\newenvironment{problem}[2][Problem]{\begin{trivlist}
\item[\hskip \labelsep {\bfseries #1}\hskip \labelsep {\bfseries #2.}]}{\end{trivlist}
\vspace{1em}
}

%problems list environment
\newlist{subprob}{enumerate}{1}
\setlist[subprob]{label={\bfseries\itshape\alph*.}}

%solution environment
\newenvironment{solution}[1][Solution]{
    \setcounter{nlem}{0}
    \vspace{-2ex}
  \begin{proof}[#1]
}{
  \phantom\qedhere\end{proof}
}


% colored boxes
\newenvironment{answerbox}{
      \begin{mdframed}[backgroundcolor=green!30, innertopmargin=12px, topline=false,rightline=false,leftline=false,bottomline=false]}{
  \end{mdframed}%
}

\newenvironment{thmbox}{
        \vskip2ex
          \begin{mdframed}[backgroundcolor=blue!10, topline=false,rightline=false,leftline=false,bottomline=false]}{
  \end{mdframed}
}

\newenvironment{algbox}[1]{
        \vskip2ex
            \begin{mdframed}[backgroundcolor=yellow!10, topline=false,rightline=false,leftline=false,bottomline=false, userdefinedwidth=.7\linewidth, align=center]
              \vspace{1ex}
              \hrulefill
              \vspace{-1.5ex}
  
              {\bf #1}
  
              \vspace{-2.5ex}
              \hrulefill
              }{
            \end{mdframed}
}

\newenvironment{algboxw}[2]{
        \vskip2ex
            \begin{mdframed}[backgroundcolor=yellow!10, topline=false,rightline=false,leftline=false,bottomline=false, userdefinedwidth=#2\linewidth, align=center]
              \vspace{1ex}
              \hrulefill
              \vspace{-1.5ex}
  
              {\bf #1}
  
              \vspace{-2.5ex}
              \hrulefill
              }{
            \end{mdframed}
}

\newenvironment{algboxnt}{
        \vskip2ex
            \begin{mdframed}[backgroundcolor=yellow!10, topline=false,rightline=false,leftline=false,bottomline=false, userdefinedwidth=.7\linewidth, align=center]\vspace{1.5ex}}{
            \end{mdframed}
}

% theorem envs
\theoremstyle{definition}
\newtheorem*{lem}{Lemma}
\newtheorem*{claim}{Claim}
\newtheorem*{thm}{Theorem}
\newtheorem*{defn}{Definition}
\newtheorem*{coro}{Corollary}
\newtheorem*{rmk}{Remark}
\newtheorem*{prop}{Proposition}
\newtheorem*{example}{Example}
\newtheorem*{probstatement}{Problem}
\newtheorem{nlem}{Lemma}
\newtheorem{ndefn}[nlem]{Definition}
\newtheorem{nthm}[nlem]{Theorem}
\newtheorem{ncor}[nlem]{Corollary}
\newtheorem{nprop}[nlem]{Proposition}
\newtheorem{nrmk}[nlem]{Remark}


% Probability
\newcommand{\Prob}[1]{\mathbb{P}\left[#1\right]}
\newcommand{\Exp}[1]{\mathbf{E}\left[#1\right]}
\newcommand{\Var}[1]{\mathbf{Var}\left[#1\right]}
\newcommand{\Cov}[1]{\mathbf{Cov}[#1]}
\renewcommand{\Pr}{\Prob}
\DeclareMathOperator{\Bin}{Bin}
\DeclareMathOperator{\Pois}{Pois}
\DeclareMathOperator{\Geom}{Geom}
\DeclareMathOperator{\Normal}{\mathcal{N}}


% Misc commands
\newcommand{\sgn}{\mathrm{sgn}}
\newcommand{\poly}{\mathrm{poly}}
\newcommand{\polylog}{\mathrm{polylog}}
\newcommand{\contradiction}{\hfill$\Rightarrow\!\Leftarrow$}
\newcommand{\writer}{Andrew Koulogeorge, Eric Richardson}
\newcommand{\email}{andrew.j.koulogeorge.24@dartmouth.edu}
\newcommand{\TODO}[1]{\textcolor{red}{\bf [TODO: #1] }}
\newcommand{\note}[1]{\textcolor{blue}{\bf [Note: #1] }}
\newcommand{\alert}[1]{\textcolor{red}{\bf [#1]}}
\renewcommand{\epsilon}{\varepsilon}
\renewcommand{\phi}{\varphi}
\newcommand{\lnorm}[2]{\left\lVert#1\right\rVert_{#2}}
\DeclareMathOperator*{\argmin}{argmin}
\DeclareMathOperator*{\argmax}{argmax}
% PSet Titles
\newcommand{\normaltitle}[3]{
    \begin{center}
        {\LARGE #2}
        \vskip-1ex
        {\large \writer}
        \vskip-1ex
        {\large Date: \today}
        \vskip-1ex
        {\large Course: #1, ~#3}
        \vskip0.1ex
        \vspace{2ex}
        \hrule height 2pt
        \vspace{2ex}
    \end{center}
}

\newcommand{\fancytitle}[3]{
	{
            \rule{\textwidth}{1pt}
		\begin{center}
			{
                    \setlength{\parskip}{0mm} \setlength{\parindent}{0mm}
				{\textbf{#2} \hfill \writer}\\
				#1:~#3 \hfill Date: \today
                }
		\end{center}
            \vskip-2ex
            \rule{\textwidth}{1pt}
	}
 }
\everymath{\displaystyle}

\begin{document}

\fancytitle{COSC 34}{Problem Set 7}{Randomized Algorithms}

Collaboration Statement: Andrew Koulogeorge and Eric Richardson 

\begin{problem}{2: A Better Analysis of \textsc{Count-Sketch}}
    
\end{problem}
\begin{solution} \ \\
    In the \textsc{Count-Sketch} framework, suppose we define $H$ to be the set of elements $i$ with the $\ceil{\frac{1}{\epsilon^2}}$-largest $\textbf{f}_i$'s. Let $L := [n] \backslash H$ represent the long tail of low frequency elements. Then, define
    \begin{align*}
        ||\textbf{f}||_{tail} = \sqrt{\sum\limits_{i \in L}\textbf{f}_i^2}
    \end{align*}
    We wish to improve the bound given in class by showing that 
    \begin{align*}
        \textbf{Pr}[|\hat{\textbf{f}}_i - \textbf{f}_i| \geq \epsilon ||\textbf{f}||_{tail}] \leq \frac{1}{3}
    \end{align*}
    To begin, suppose we fix $i \in [n]$. Then, by the analysis given in class, we have that the entry in the counter table $C$ at index $h(i)$, where $h$ is drawn independently from a UHF with domain size $[n]$ and range $k$, is given by
    \begin{align*}
        C[h(i)] = g(i) \cdot \textbf{f}_i + \sum\limits_{j \neq i:h(j) = h(i)} g(j) \cdot \textbf{f}_j
    \end{align*}
    Also note that $g$ is drawn from a strongly independent UHF with domain size $[n]$ and range $\{-1, 1\}$. Then, since our estimate $\hat{\textbf{f}}_i$ is equivalent to $C[h(i)] \cdot g(i)$, we have
    \begin{align*}
        &\hat{\textbf{f}}_i = g(i) \cdot (g(i) \cdot \textbf{f}_i + \sum\limits_{j \neq i:h(j) = h(i)} g(j) \cdot \textbf{f}_j)\\
        &= \textbf{f}_i + \sum\limits_{j \neq i:h(j) = h(i)} g(i)g(j) \cdot \textbf{f}_j 
    \end{align*}
    Intuitively, this represents our ground truth value $\textbf{f}_i$ in addition to some error terms that may occur due to hash collisions. Since $H$ and $L$ are disjoint, we can split this summation over collisions into those that occur in $H$ and those that occur in $L$:
    \begin{align*}
        &\hat{\textbf{f}}_i = \textbf{f}_i + \sum\limits_{j \neq i \in H: h(j) = h(i)} g(i)g(j)\textbf{f}_j + \sum\limits_{k \neq i \in L: h(k) = h(i)} g(i)g(k)\textbf{f}_k \\
        \implies &\hat{\textbf{f}}_i - \textbf{f}_i = \sum\limits_{j \neq i \in H: h(j) = h(i)} g(i)g(j)\textbf{f}_j + \sum\limits_{k \neq i \in L: h(k) = h(i)} g(i)g(k)\textbf{f}_k
    \end{align*}
    Let $A$ be a random variable that takes the value of the summation over $H$, and $B$ a random variable that takes the value of the summation over $L$. First, note that $\textbf{Exp}[A] = \textbf{Exp}[B] = 0$ due to the fact that $\textbf{Exp}[g(i)g(j)] = 0$ for any $j \neq i$. We now calculate the variance of $B$:
    \begin{align*}
        \textbf{Var}[B] &= \textbf{Exp}[B^2] - \textbf{Exp}[B]^2 \\
        &= \textbf{Exp}[B^2] \\
        &= \sum\limits_{j \neq i \in H} \textbf{Pr}[h(j) = h(i)] \cdot g(i)^2g(j)^2\textbf{f}_j^2 \\
        &\leq \sum\limits_{j \neq i \in H} \frac{1}{k} \cdot \textbf{f}_j^2 \\
        &= \frac{||f||_{tail}^2}{k} \\ 
    \end{align*}
    Then, applying Chebyshev with $t = \epsilon ||\textbf{f}||_{tail}$, we have
    \begin{align*}
        \textbf{Pr}[|B| \geq \epsilon ||\textbf{f}||_{tail}] &\leq \frac{||\textbf{f}||_{tail}^2}{k\epsilon^2||\textbf{f}||_{tail}^2} \\
        &= \frac{1}{10}
    \end{align*}
    where the last step follows with $k = \frac{10}{\epsilon^2}$. Thus, we have bounded the probability that the magnitude of error accumulated by collisions from $L$ exceeds $\epsilon||\textbf{f}||_{tail}$. Now, consider $A$. Trivially, the probability that $A \geq \epsilon ||\textbf{f}||_{tail}$ can be no greater than the probability that some element in $H$ collides with $i$. More generally, we have that
    \begin{align*}
        \textbf{Pr}[|A| \geq \epsilon||\textbf{f}||_{tail}] \leq \textbf{Pr}[A \neq 0]
    \end{align*}
    Since the only case in which $A$ is nonzero is when there exists some element $j \in H$ such that $h(j) = h(i)$, we can apply a union bound over the probability that a hash collision occurs in $H$:
    \begin{align*}
        \textbf{Pr}[A \neq 0] &\leq \textbf{Pr}[\exists j \in H: h(j) = h(i)] \\
        &\leq \sum\limits_{j \in H} \frac{1}{k} \\
        &= \frac{1}{k\epsilon^2} \\
        &= \frac{1}{10}
    \end{align*}
    Therefore, we have that $\textbf{Pr}[|A| \geq \epsilon||\textbf{f}||_{tail}]$ and $\textbf{Pr}[|B| \geq \epsilon||\textbf{f}||_{tail}]$ are both at most $\frac{1}{10}$. Now, let $\mathcal{E}$ denote the event that $|\hat{\textbf{f}}_i - \textbf{f}_i| \geq \epsilon||\textbf{f}||_{tail}$. Suppose $A = 0$, then the only case in which $\mathcal{E}$ occurs is when $|B| \geq \epsilon||\textbf{f}||_{tail}$. Similarly, suppose $|B| < \epsilon||\textbf{f}||_{tail}$. Then, it is impossible for $\mathcal{E}$ to occur unless $A \neq 0$. Thus, it follows that
    \begin{align*}
        \textbf{Pr}[\mathcal{E}] &\leq \textbf{Pr}[A \neq 0 \lor |B| \geq \epsilon||\textbf{f}||_{tail}] \\
        &\leq \textbf{Pr}[A \neq 0] + \textbf{Pr}[|B| \geq \epsilon||\textbf{f}||_{tail}] \\
        &= \frac{1}{10} + \frac{1}{10} \\
        &= \frac{1}{5}
    \end{align*}
    Hence, \textsc{Count-Sketch} with $k \leq \frac{10}{\epsilon^2}$ satisfies 
    \begin{align*}
          \textbf{Pr}[|\hat{\textbf{f}}_i - \textbf{f}_i| \geq \epsilon ||\textbf{f}||_{tail}] \leq \frac{1}{3}
    \end{align*}
    for every $i \in [n]$.
\end{solution}



\end{document}

