\documentclass[12pt]{article}
\usepackage{lipsum}
\usepackage{ifthen}
\usepackage{tikz-cd}
\usepackage{soul}
\usepackage{fullpage, prettyref}
\usepackage{fancyhdr}
\usepackage{palatino}
\usepackage[T1]{fontenc}
\usepackage{ulem}
\usepackage{graphicx}
\usepackage{ifthen}
\usepackage{algorithm}

\usepackage{a4}
\usepackage[margin=1in]{geometry} 
\usepackage[noend]{algpseudocode}
\usepackage{amsmath, amssymb, amsthm, mathtools}
\usepackage{setspace}
\usepackage{multicol}
\usepackage{array}
\usepackage{enumitem}
\usepackage{csquotes}
\usepackage{url}
\usepackage{boxedminipage}
\usepackage{wrapfig}
\usepackage{color, xcolor}
\usepackage{transparent}
\usepackage{parskip}
\usepackage{varwidth}
\usepackage{bm}
\usepackage{xspace}
\usepackage{url}
\usepackage{fullpage, prettyref}
\usepackage{boxedminipage}
\usepackage{wrapfig}
\usepackage{ifthen}
\usepackage{framed}
\usepackage{mdframed}
\usepackage[colorlinks=true,urlcolor=blue,linkcolor=blue,citecolor=violet,pdfstartview=FitH]{hyperref}
\usepackage[noabbrev,nameinlink]{cleveref}
\usepackage{enumitem}
\usepackage{tikz}
\usepackage{mathrsfs} 
\usepackage{bbm}
\usepackage[backend=biber,bibencoding=ascii,style=alphabetic,sorting=none]{biblatex}

% config
\renewcommand{\headrulewidth}{0pt}
\renewcommand{\footrulewidth}{0pt}
\pagestyle{fancy}
\fancyhead{}
\fancyfoot{}
\fancyfoot[R]{\thepage}

% Algebra
\newcommand{\unit}[1]{#1^{\times}}
\DeclareMathOperator{\Orb}{Orb}
\DeclareMathOperator{\Stab}{Stab}
\DeclareMathOperator{\Hom}{Hom}
\DeclareMathOperator{\End}{End}
\DeclareMathOperator{\Aut}{Aut}
\DeclareMathOperator{\Gal}{Gal}
\DeclareMathOperator{\rank}{rk}
\DeclareMathOperator{\img}{img}
\DeclareMathOperator{\id}{id}
\DeclareMathOperator{\ch}{ch}
\DeclareMathOperator{\tr}{tr}
\DeclareMathOperator{\codim}{codim}
\DeclareMathOperator{\ann}{ann}
\DeclareMathOperator{\ev}{ev}
\DeclareMathOperator{\proj}{proj}
\DeclareMathOperator{\ang}{ang}
\DeclareMathOperator{\GL}{GL}
\DeclareMathOperator{\Ob}{Ob}
\newcommand{\category}[1]{\mathsf{\underline{#1}}}
\newcommand{\acts}{\circlearrowleft}
\newcommand{\tleq}{\unlhd}

% Boolean logic/algorithms
\newcommand{\cor}{\textsc{ or }}
\newcommand{\cand}{\textsc{ and }}
\newcommand{\xor}{\oplus}
\newcommand{\true}{\textsc{True}}
\newcommand{\false}{\textsc{False}}
\newcommand{\nil}{\textsc{Nil}}
\newcommand{\alg}[1]{\textsc{#1}}
\DeclareMathOperator{\cmod}{mod}
\newcommand{\eq}{\leftarrow}
\newcommand{\chapterref}[2]{[\S #1, #2]}

% math formatting
\def\bar{\overline}
\newcommand{\script}[1]{\mathcal{#1}}
\newcommand{\fancyscript}[1]{\mathscr{#1}}


% number sets
\newcommand{\N}{\mathbb{N}}
\newcommand{\Q}{\mathbb{Q}}
\newcommand{\R}{\mathbb{R}}
\newcommand{\Z}{\mathbb{Z}}
\newcommand{\C}{\mathbb{C}}
\newcommand{\F}{\mathbb{F}}

%Paired Delims
\def\set#1{\left\{ #1 \right\}}
\def\angle#1{\left\langle #1 \right\rangle}
\def\sqbrackets#1{\left[ #1 \right]}
\def\brackets#1{\left( #1 \right)}
\DeclarePairedDelimiter\ceil{\lceil}{\rceil}
\DeclarePairedDelimiter\floor{\lfloor}{\rfloor}
\newcommand{\abs}[1]{\left\lvert #1 \right\rvert}

%problem environment
\newenvironment{problem}[2][Problem]{\begin{trivlist}
\item[\hskip \labelsep {\bfseries #1}\hskip \labelsep {\bfseries #2.}]}{\end{trivlist}
\vspace{1em}
}

%problems list environment
\newlist{subprob}{enumerate}{1}
\setlist[subprob]{label={\bfseries\itshape\alph*.}}

%solution environment
\newenvironment{solution}[1][Solution]{
    \setcounter{nlem}{0}
    \vspace{-2ex}
  \begin{proof}[#1]
}{
  \phantom\qedhere\end{proof}
}


% colored boxes
\newenvironment{answerbox}{
      \begin{mdframed}[backgroundcolor=green!30, innertopmargin=12px, topline=false,rightline=false,leftline=false,bottomline=false]}{
  \end{mdframed}%
}

\newenvironment{thmbox}{
        \vskip2ex
          \begin{mdframed}[backgroundcolor=blue!10, topline=false,rightline=false,leftline=false,bottomline=false]}{
  \end{mdframed}
}

\newenvironment{algbox}[1]{
        \vskip2ex
            \begin{mdframed}[backgroundcolor=yellow!10, topline=false,rightline=false,leftline=false,bottomline=false, userdefinedwidth=.7\linewidth, align=center]
              \vspace{1ex}
              \hrulefill
              \vspace{-1.5ex}
  
              {\bf #1}
  
              \vspace{-2.5ex}
              \hrulefill
              }{
            \end{mdframed}
}

\newenvironment{algboxw}[2]{
        \vskip2ex
            \begin{mdframed}[backgroundcolor=yellow!10, topline=false,rightline=false,leftline=false,bottomline=false, userdefinedwidth=#2\linewidth, align=center]
              \vspace{1ex}
              \hrulefill
              \vspace{-1.5ex}
  
              {\bf #1}
  
              \vspace{-2.5ex}
              \hrulefill
              }{
            \end{mdframed}
}

\newenvironment{algboxnt}{
        \vskip2ex
            \begin{mdframed}[backgroundcolor=yellow!10, topline=false,rightline=false,leftline=false,bottomline=false, userdefinedwidth=.7\linewidth, align=center]\vspace{1.5ex}}{
            \end{mdframed}
}

% theorem envs
\theoremstyle{definition}
\newtheorem*{lem}{Lemma}
\newtheorem*{claim}{Claim}
\newtheorem*{thm}{Theorem}
\newtheorem*{defn}{Definition}
\newtheorem*{coro}{Corollary}
\newtheorem*{rmk}{Remark}
\newtheorem*{prop}{Proposition}
\newtheorem*{example}{Example}
\newtheorem*{probstatement}{Problem}
\newtheorem{nlem}{Lemma}
\newtheorem{ndefn}[nlem]{Definition}
\newtheorem{nthm}[nlem]{Theorem}
\newtheorem{ncor}[nlem]{Corollary}
\newtheorem{nprop}[nlem]{Proposition}
\newtheorem{nrmk}[nlem]{Remark}


% Probability
\newcommand{\Prob}[1]{\mathbb{P}\left[#1\right]}
\newcommand{\Exp}[1]{\mathbf{E}\left[#1\right]}
\newcommand{\Var}[1]{\mathbf{Var}\left[#1\right]}
\newcommand{\Cov}[1]{\mathbf{Cov}[#1]}
\renewcommand{\Pr}{\Prob}
\DeclareMathOperator{\Bin}{Bin}
\DeclareMathOperator{\Pois}{Pois}
\DeclareMathOperator{\Geom}{Geom}
\DeclareMathOperator{\Normal}{\mathcal{N}}


% Misc commands
\newcommand{\sgn}{\mathrm{sgn}}
\newcommand{\poly}{\mathrm{poly}}
\newcommand{\polylog}{\mathrm{polylog}}
\newcommand{\contradiction}{\hfill$\Rightarrow\!\Leftarrow$}
\newcommand{\writer}{Andrew Koulogeorge, Eric Richardson}
\newcommand{\email}{andrew.j.koulogeorge.24@dartmouth.edu}
\newcommand{\TODO}[1]{\textcolor{red}{\bf [TODO: #1] }}
\newcommand{\note}[1]{\textcolor{blue}{\bf [Note: #1] }}
\newcommand{\alert}[1]{\textcolor{red}{\bf [#1]}}
\renewcommand{\epsilon}{\varepsilon}
\renewcommand{\phi}{\varphi}
\newcommand{\lnorm}[2]{\left\lVert#1\right\rVert_{#2}}
\DeclareMathOperator*{\argmin}{argmin}
\DeclareMathOperator*{\argmax}{argmax}
% PSet Titles
\newcommand{\normaltitle}[3]{
    \begin{center}
        {\LARGE #2}
        \vskip-1ex
        {\large \writer}
        \vskip-1ex
        {\large Date: \today}
        \vskip-1ex
        {\large Course: #1, ~#3}
        \vskip0.1ex
        \vspace{2ex}
        \hrule height 2pt
        \vspace{2ex}
    \end{center}
}

\newcommand{\fancytitle}[3]{
	{
            \rule{\textwidth}{1pt}
		\begin{center}
			{
                    \setlength{\parskip}{0mm} \setlength{\parindent}{0mm}
				{\textbf{#2} \hfill \writer}\\
				#1:~#3 \hfill Date: \today
                }
		\end{center}
            \vskip-2ex
            \rule{\textwidth}{1pt}
	}
 }
\everymath{\displaystyle}

\begin{document}

\fancytitle{COSC 34}{Problem Set 2}{Randomized Algorithms}

Collaboration Statement: Andrew Koulogeorge and Eric Richardson 

\begin{problem}{5: The better way of estimating bias of an unknown coin}
\end{problem}
\begin{solution} 
Note that the total number of tosses needed to get $N$ heads can be expressed as the number of tosses between the $i-1$th and $i$th head. Thus, we can compute $\Exp{Z} = \sum_{i=1}^{N}{Z_i} = \frac{N}{p^*}$ since each $Z_i$ is a geometric random variable with probability $p^*$ of succeeding $\implies \Exp{Z_i} = \frac{1}{p^*}$. \qed \\
In order to prove that the probability that the number of tosses needed to get $N$ heads concentrates "close" to $\frac{N}{p^*}$, we need to connect the events related to $Z$ to events related to another random variable $X$ and $Y$ which we can represent as sums of Bernoulli random variables and thus bound their probability of deviating from the mean with Chernoff bounds. Note that its impossible to represent $Z$ itself as a sum of a finite number of Bernoulli random variables since the image of $Z$ is infinite. \\

Let A be the event that $Z < (1-\epsilon)\frac{N}{p^*}$ and let B be the event that $Z > (1+\epsilon)\frac{N}{p^*}$. Note that we wish to bound $\Pr{A\cup B} = \Pr{A} + \Pr{B}$ since A and B are disjoint events. Note that the union of these two events is identical to the event that $Z$ does not lie within $\epsilon$ of its mean.\\

Let $X$ be the number of heads in $\frac{(1+ \epsilon)N}{p^*}$ tosses and let $Y$ be the number of heads in $\frac{(1-\epsilon)N}{p^*}$ tosses and let $Y$ be the number of heads in $\frac{(1+ \epsilon)N}{p^*}$ tosses. We can see that $\Exp{X} = (1+\epsilon)N$ and $\Exp{Y} = (1-\epsilon)N$ since they are sums of Bernoulli random variables and we can again apply LoE.\\

Now for the crux of the argument; I claim that the probability it takes less than $(1-\epsilon)\frac{N}{p^*}$ tosses to see $N$ is the same as the probability that we get at least $N$ coins in $(1-\epsilon)\frac{N}{p^*}$ tosses. The same argument applies for the over estimate case: the probability it takes more than $(1+\epsilon)\frac{N}{p^*}$ tosses to see $N$ is the same as the probability that we get less than $N$ coins in $(1+\epsilon)\frac{N}{p^*}$ tosses.
\[
\textbf{Pr}[A] = \textbf{Pr}[Z < (1-\epsilon)\frac{N}{p^*}] = \textbf{Pr}[Y > N]
\]
\[
\textbf{Pr}[B] = \textbf{Pr}[Z > (1+\epsilon)\frac{N}{p^*}] = \textbf{Pr}[X <  N]
\]

Since $X$ and $Y$ are both sums of independent Bernoulli random variables, we can apply Chernoff. Let $t_x = \frac{\epsilon}{1 + \epsilon}$ and $t_y = \frac{\epsilon}{1 - \epsilon}$:

\[
\textbf{Pr}[A] = \textbf{Pr}[Y > N] = \textbf{Pr}[Y > (1+t_y)\Exp{Y}] \leq \exp{\frac{t_{y}^2\Exp{X}}{3}} = \exp{{\frac{-\epsilon^2N}{3(1 -\epsilon)}}} < \exp{{\frac{-\epsilon^2N}{3}}}
\]
\[
\textbf{Pr}[B] = \textbf{Pr}[X <  N] = \textbf{Pr}[X < (1-t_x)\Exp{X}] \leq \exp{\frac{t_{x}^2\Exp{X}}{2}} = \exp{{\frac{-\epsilon^2N}{2(1 + \epsilon)}}} < \exp{{\frac{-\epsilon^2N}{3}}}
\]
where both of the last inequalities follow from the fact that $\epsilon \in (0,\frac{1}{2})$. Thus, we have shown that 

\[
\textbf{Pr}[Z \notin (1 \pm \epsilon)\Exp{Z}]=  \textbf{Pr}[A] + \textbf{Pr}[B] < 2\exp{{\frac{-\epsilon^2N}{3}}}
\]
\qed\\
Qualitatively, we have now shown that if we flip our coin until we get more than $N=\frac{3}{\epsilon^2}\log{\frac{2}{\delta}} $ heads, then the number of tosses it will take will concentrate around $\Exp{Z} = \frac{N}{p^*}$ w.h.p. Note that this holds $\forall~\epsilon~\in~(0,\frac{1}{2})$. In our algorithm, we are going to restrict $\forall~\epsilon~\in~(0,\frac{1}{4})$. Note that our result in part b still holds for this restricted range of values on $\epsilon$. We can use this bound to now construct an estimator for $p^*$ which will also concentrate around the true value of $p^*$ w.h.p.

\begin{answerbox}
\begin{algorithmic}
\Procedure{Estimate $p^*$}{$\epsilon_0$, $\delta$}
\State Assert $\epsilon_0~\in~(0,\frac{1}{3})$
\State $\epsilon = \frac{\epsilon_0}{1 + \epsilon_0}$ (Always less than $\frac{1}{2} \implies$ bound holds)
\State Let $N > \frac{3}{\epsilon^2}\log{\frac{2}{\delta}}$

\State Flip coin until $N$ heads appear, keeping a count of $\textbf{numFlips}$

\State $\hat{p} = \frac{N}{numFlips}$

\State \Return{$\hat{p}$}

\EndProcedure
\end{algorithmic}
\end{answerbox}
By flipping our coin until we see greater than $N = \frac{3}{\epsilon^2}\log{\frac{2}{\delta}}$ heads, we know:
\[
\textbf{Pr}[Z \in (1 \pm \epsilon)\frac{N}{p^*}] \geq 1- \delta
\]
Lets define our estimate $\hat{p} = \frac{N}{Z}$. We know that with probability at least $1-\delta$ (something large)
\[
\implies Z \leq (1 + \epsilon)\frac{N}{p^*} ~~~\&~~~ Z \geq (1 - \epsilon)\frac{N}{p^*}
\]
\[
\implies p^* \leq (1 + \epsilon)\hat{p} ~~~\&~~~ p^* \geq (1 - \epsilon)\hat{p}
\]
\[
\implies \frac{1}{1 + \epsilon}p^* \leq \hat{p} ~~~\&~~~ \frac{1}{1 - \epsilon}p^* \geq \hat{p}
\]
Note that the above inequalities hold $\forall~\epsilon~\in~(0,\frac{1}{2})$. Lets choose $\epsilon \in (0,\frac{1}{4})$ and define $\epsilon_0 = \frac{\epsilon}{1-\epsilon}$. NOTE that we have a degree of freedom over $\epsilon$ here because we proved a result about $Z$ which holds true $\forall~\epsilon~\in~(0,\frac{1}{2})$ and so it also will hold true for a smaller subset of values for $\epsilon$. With this clever construction of $\epsilon_0$, we see that $\epsilon_0$ lies between $(0,\frac{1}{3})$:
\[
\epsilon = 0 \implies \epsilon_0 = \frac{\epsilon}{1 - \epsilon} = 0
\]
\[
\epsilon = \frac{1}{4} \implies \epsilon_0 = \frac{\epsilon}{1 - \epsilon} = \frac{1}{3}
\]
Note that $\epsilon$ cant equal $0$ or $\frac{1}{4}$ exactly but this analysis shows that $\epsilon_0 \in (0,\frac{1}{3})$. Next, we see that when we restrict $\epsilon \in (0,\frac{1}{4})$, we are able to bound the region where our estimate $\hat{p}$ will lie:

\[
(1-\epsilon_0) \leq \frac{1}{1 + \epsilon}
\]
\[
\epsilon = 0 \implies 1 \leq 1
\]
\[
\epsilon = \frac{1}{4} \implies \frac{2}{3} \leq \frac{4}{5}
\]
\[
(1+\epsilon_0) \geq \frac{1}{1 - \epsilon}
\]
\[
\epsilon = 0 \implies 1 \geq 1
\]
\[
\epsilon = \frac{1}{4} \implies \frac{4}{3} \geq \frac{4}{3}
\]
Since $p^* > 0$, we can multiply these inequalities and preserve the ordering:
\[
(1-\epsilon_0)p^* \leq \frac{1}{1 + \epsilon}p^* \leq \hat{p}
\]
\[
(1+\epsilon_0)p^* \geq \frac{1}{1 - \epsilon}p^* \geq \hat{p}
\]
Since the bounds on $Z$ hold with probability $1-\delta$, our estimate $\hat{p}$ falls in this range as well for any $\epsilon_0\in (0,\frac{1}{3})$

\[
\implies (1-\epsilon_0)p^* \leq \hat{p} \leq (1+\epsilon_0)p^* \qed
\]


\end{solution}
\end{document}