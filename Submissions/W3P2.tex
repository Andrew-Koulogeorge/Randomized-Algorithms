\documentclass[12pt]{article}
\usepackage{lipsum}
\usepackage{ifthen}
\usepackage{tikz-cd}
\usepackage{soul}
\usepackage{fullpage, prettyref}
\usepackage{fancyhdr}
\usepackage{palatino}
\usepackage[T1]{fontenc}
\usepackage{ulem}
\usepackage{graphicx}
\usepackage{ifthen}
\usepackage{algorithm}

\usepackage{a4}
\usepackage[margin=1in]{geometry} 
\usepackage[noend]{algpseudocode}
\usepackage{amsmath, amssymb, amsthm, mathtools}
\usepackage{setspace}
\usepackage{multicol}
\usepackage{array}
\usepackage{enumitem}
\usepackage{csquotes}
\usepackage{url}
\usepackage{boxedminipage}
\usepackage{wrapfig}
\usepackage{color, xcolor}
\usepackage{transparent}
\usepackage{parskip}
\usepackage{varwidth}
\usepackage{bm}
\usepackage{xspace}
\usepackage{url}
\usepackage{fullpage, prettyref}
\usepackage{boxedminipage}
\usepackage{wrapfig}
\usepackage{ifthen}
\usepackage{framed}
\usepackage{mdframed}
\usepackage[colorlinks=true,urlcolor=blue,linkcolor=blue,citecolor=violet,pdfstartview=FitH]{hyperref}
\usepackage[noabbrev,nameinlink]{cleveref}
\usepackage{enumitem}
\usepackage{tikz}
\usepackage{mathrsfs} 
\usepackage{bbm}
\usepackage[backend=biber,bibencoding=ascii,style=alphabetic,sorting=none]{biblatex}

% config
\renewcommand{\headrulewidth}{0pt}
\renewcommand{\footrulewidth}{0pt}
\pagestyle{fancy}
\fancyhead{}
\fancyfoot{}
\fancyfoot[R]{\thepage}

% Algebra
\newcommand{\unit}[1]{#1^{\times}}
\DeclareMathOperator{\Orb}{Orb}
\DeclareMathOperator{\Stab}{Stab}
\DeclareMathOperator{\Hom}{Hom}
\DeclareMathOperator{\End}{End}
\DeclareMathOperator{\Aut}{Aut}
\DeclareMathOperator{\Gal}{Gal}
\DeclareMathOperator{\rank}{rk}
\DeclareMathOperator{\img}{img}
\DeclareMathOperator{\id}{id}
\DeclareMathOperator{\ch}{ch}
\DeclareMathOperator{\tr}{tr}
\DeclareMathOperator{\codim}{codim}
\DeclareMathOperator{\ann}{ann}
\DeclareMathOperator{\ev}{ev}
\DeclareMathOperator{\proj}{proj}
\DeclareMathOperator{\ang}{ang}
\DeclareMathOperator{\GL}{GL}
\DeclareMathOperator{\Ob}{Ob}
\newcommand{\category}[1]{\mathsf{\underline{#1}}}
\newcommand{\acts}{\circlearrowleft}
\newcommand{\tleq}{\unlhd}

% Boolean logic/algorithms
\newcommand{\cor}{\textsc{ or }}
\newcommand{\cand}{\textsc{ and }}
\newcommand{\xor}{\oplus}
\newcommand{\true}{\textsc{True}}
\newcommand{\false}{\textsc{False}}
\newcommand{\nil}{\textsc{Nil}}
\newcommand{\alg}[1]{\textsc{#1}}
\DeclareMathOperator{\cmod}{mod}
\newcommand{\eq}{\leftarrow}
\newcommand{\chapterref}[2]{[\S #1, #2]}

% math formatting
\def\bar{\overline}
\newcommand{\script}[1]{\mathcal{#1}}
\newcommand{\fancyscript}[1]{\mathscr{#1}}


% number sets
\newcommand{\N}{\mathbb{N}}
\newcommand{\Q}{\mathbb{Q}}
\newcommand{\R}{\mathbb{R}}
\newcommand{\Z}{\mathbb{Z}}
\newcommand{\C}{\mathbb{C}}
\newcommand{\F}{\mathbb{F}}

%Paired Delims
\def\set#1{\left\{ #1 \right\}}
\def\angle#1{\left\langle #1 \right\rangle}
\def\sqbrackets#1{\left[ #1 \right]}
\def\brackets#1{\left( #1 \right)}
\DeclarePairedDelimiter\ceil{\lceil}{\rceil}
\DeclarePairedDelimiter\floor{\lfloor}{\rfloor}
\newcommand{\abs}[1]{\left\lvert #1 \right\rvert}

%problem environment
\newenvironment{problem}[2][Problem]{\begin{trivlist}
\item[\hskip \labelsep {\bfseries #1}\hskip \labelsep {\bfseries #2.}]}{\end{trivlist}
\vspace{1em}
}

%problems list environment
\newlist{subprob}{enumerate}{1}
\setlist[subprob]{label={\bfseries\itshape\alph*.}}

%solution environment
\newenvironment{solution}[1][Solution]{
    \setcounter{nlem}{0}
    \vspace{-2ex}
  \begin{proof}[#1]
}{
  \phantom\qedhere\end{proof}
}


% colored boxes
\newenvironment{answerbox}{
      \begin{mdframed}[backgroundcolor=green!30, innertopmargin=12px, topline=false,rightline=false,leftline=false,bottomline=false]}{
  \end{mdframed}%
}

\newenvironment{thmbox}{
        \vskip2ex
          \begin{mdframed}[backgroundcolor=blue!10, topline=false,rightline=false,leftline=false,bottomline=false]}{
  \end{mdframed}
}

\newenvironment{algbox}[1]{
        \vskip2ex
            \begin{mdframed}[backgroundcolor=yellow!10, topline=false,rightline=false,leftline=false,bottomline=false, userdefinedwidth=.7\linewidth, align=center]
              \vspace{1ex}
              \hrulefill
              \vspace{-1.5ex}
  
              {\bf #1}
  
              \vspace{-2.5ex}
              \hrulefill
              }{
            \end{mdframed}
}

\newenvironment{algboxw}[2]{
        \vskip2ex
            \begin{mdframed}[backgroundcolor=yellow!10, topline=false,rightline=false,leftline=false,bottomline=false, userdefinedwidth=#2\linewidth, align=center]
              \vspace{1ex}
              \hrulefill
              \vspace{-1.5ex}
  
              {\bf #1}
  
              \vspace{-2.5ex}
              \hrulefill
              }{
            \end{mdframed}
}

\newenvironment{algboxnt}{
        \vskip2ex
            \begin{mdframed}[backgroundcolor=yellow!10, topline=false,rightline=false,leftline=false,bottomline=false, userdefinedwidth=.7\linewidth, align=center]\vspace{1.5ex}}{
            \end{mdframed}
}

% theorem envs
\theoremstyle{definition}
\newtheorem*{lem}{Lemma}
\newtheorem*{claim}{Claim}
\newtheorem*{thm}{Theorem}
\newtheorem*{defn}{Definition}
\newtheorem*{coro}{Corollary}
\newtheorem*{rmk}{Remark}
\newtheorem*{prop}{Proposition}
\newtheorem*{example}{Example}
\newtheorem*{probstatement}{Problem}
\newtheorem{nlem}{Lemma}
\newtheorem{ndefn}[nlem]{Definition}
\newtheorem{nthm}[nlem]{Theorem}
\newtheorem{ncor}[nlem]{Corollary}
\newtheorem{nprop}[nlem]{Proposition}
\newtheorem{nrmk}[nlem]{Remark}


% Probability
\newcommand{\Prob}[1]{\mathbb{P}\left[#1\right]}
\newcommand{\Exp}[1]{\mathbf{E}\left[#1\right]}
\newcommand{\Var}[1]{\mathbf{Var}\left[#1\right]}
\newcommand{\Cov}[1]{\mathbf{Cov}[#1]}
\renewcommand{\Pr}{\Prob}
\DeclareMathOperator{\Bin}{Bin}
\DeclareMathOperator{\Pois}{Pois}
\DeclareMathOperator{\Geom}{Geom}
\DeclareMathOperator{\Normal}{\mathcal{N}}


% Misc commands
\newcommand{\sgn}{\mathrm{sgn}}
\newcommand{\poly}{\mathrm{poly}}
\newcommand{\polylog}{\mathrm{polylog}}
\newcommand{\contradiction}{\hfill$\Rightarrow\!\Leftarrow$}
\newcommand{\writer}{Andrew Koulogeorge, Eric Richardson}
\newcommand{\email}{andrew.j.koulogeorge.24@dartmouth.edu}
\newcommand{\TODO}[1]{\textcolor{red}{\bf [TODO: #1] }}
\newcommand{\note}[1]{\textcolor{blue}{\bf [Note: #1] }}
\newcommand{\alert}[1]{\textcolor{red}{\bf [#1]}}
\renewcommand{\epsilon}{\varepsilon}
\renewcommand{\phi}{\varphi}
\newcommand{\lnorm}[2]{\left\lVert#1\right\rVert_{#2}}
\DeclareMathOperator*{\argmin}{argmin}
\DeclareMathOperator*{\argmax}{argmax}
% PSet Titles
\newcommand{\normaltitle}[3]{
    \begin{center}
        {\LARGE #2}
        \vskip-1ex
        {\large \writer}
        \vskip-1ex
        {\large Date: \today}
        \vskip-1ex
        {\large Course: #1, ~#3}
        \vskip0.1ex
        \vspace{2ex}
        \hrule height 2pt
        \vspace{2ex}
    \end{center}
}

\newcommand{\fancytitle}[3]{
	{
            \rule{\textwidth}{1pt}
		\begin{center}
			{
                    \setlength{\parskip}{0mm} \setlength{\parindent}{0mm}
				{\textbf{#2} \hfill \writer}\\
				#1:~#3 \hfill Date: \today
                }
		\end{center}
            \vskip-2ex
            \rule{\textwidth}{1pt}
	}
 }
\everymath{\displaystyle}

\begin{document}

\fancytitle{COSC 34}{Problem Set 3}{Randomized Algorithms}

Collaboration Statement: Andrew Koulogeorge and Eric Richardson 

\begin{problem}{2: Probabilistic Version of \#DNF}
\end{problem}
\begin{solution} \ \\
Let $x^*$ be a random vector where each entry $x^*_i$ of $x^*$ is sampled with probability $p_i$. Its important to note that our sample space is over all bit vectors of length $n$ where each bit vector has a non uniform probability of being sampled. That is, each vector has a probability mass associated with it which equals the product of the probability of each entry occurring. Also note then that $p^* = \Pr{\phi(x^*)=1}$ where $\phi(x^*)$ is $1$ if $x^*$ satisfies at least one of the clauses in the DNF formula and $0$ otherwise. Now, let us observe the following key insights:
\begin{itemize}
    \item The probability that $x^*$ satisfies a clause $C_i$ in the DNF formula can be computed in linear time. We can loop over the variables in $C_i$ and keep a running product; for each literal $x_i$, if its negated we multiply our running probability by $(1-p_i)$ and otherwise by $p_i$. Thus, we can compute the probability that $x^*$ satisfies $C_i ~\forall~i$ in $O(mn)$ time where $n$ is the number of variables and $m$ is the number of clauses in our DNF. Let this probability for a particular clause $C_i$ be denoted $q_i$
    \item Let $Q_i$ be the event that $x^*$ satisfies clause $C_i$. It then follows $x^*$ satisfies our DNF formula when it satisfies at least one of the clauses in the DNF: $p^* = \Pr{\phi(x^*)=1} = \Pr{\bigcup_{i=1}^m{Q_i}}$. By union bound, we can compute an upper bound for $p^*$: $p^* \leq \sum_{i=1}^m{\Pr{Q_i}} = \sum_{i=1}^m{q_i} = q$. 
\end{itemize}
First, its very important to note what probability space these probabilities are over: they are over the space of all vectors $x$ sampled with the probability masses $p_1 \dots p_n$. Later in the problem we are going to see another probability symbol which is over different space. For ease, lets define the probability a vector $x$ is sampled in this space as $w(x)$.\\

Second, recall that the union bound can be a gross overestimate of the probability of a union of events occurring because it double counts the probability of events which occur simultaneously. Thus, the more clauses that a given $x^*$ satisfies, the more of an overestimate $q$ will be to $p$ and our algorithm will have to take this into account. From here, we can construct an algorithm for an unbiased estimate $\hat{p}$ for $p^*$ which has low variance. The below algorithm follows very closely the sampling algorithm seen in class for estimating \#DNF. \\
\newpage
\begin{answerbox}
\begin{algorithmic}
\Procedure{Sampler $p^*$}{}
\State Compute upper bound on $p^* \rightarrow q = \sum_{i=1}^m{q_i}$
\State Sample clause $C_i$ with probability $\frac{q_i}{q}$
\State Sample $y$ s.t is satisfies $C_i$ and sample the remaining literals $y_i$ from $p_i$.
\State Compute the number of clauses that $y$ satisfies $\rightarrow ~ N(y)$
\State Compute estimator $\rightarrow ~ \hat{p} = \frac{q}{N(y)}$
\State \Return{$\hat{p}$}

\EndProcedure
\end{algorithmic}
\end{answerbox}
First, I claim that $\hat{p}$ is an unbiased estimate for $p^*$. Let U be the set of bit vectors $x$ of length $n$ such that $\phi(x) = 1$, let $Sampler(y)$ be the probability that our algorithm samples bit vector $y$ and $\hat{p}(y)$ be the value of our estimator (which is a random variable and thus a map from our sample space containing bit vectors to $\R$) for a given sample $y$. 
\[
\Exp{\hat{p}} = \sum_{y\in U}{\Pr{Sampler(y)}*\hat{p}(y)}
\]
Note that this expectation is over a new sample space; that is, the probability our algorithm samples a vector $y$ is not the same as the probability that $x^*$ is sampled. Indeed, our sampler never samples a vector which does not satisfy the DNF formula. Therefore, our algorithm induces a \textbf{new probability distribution} over the which vectors can be sampled. Now, lets take a closer look at for a fixed vector $y\in U$ what the probability our algorithm samples $y$. Let us apply the law of total probability and condition on the event of which clause $C_i$ we sample. Conditioned on which clause we sample, we can then compute the probability that $y$ is sampled. Note that we need only consider the clauses $C_j$ which $y$ is satisfies since we impose that constraint before we sample the remaining values for $y$. We denote the number of these clauses that $y$ satisfies at $N(y)$
\[
\Pr{Sampler(y)} = \sum_{C_j}{\Pr{Sampler(C_j)}\Pr{Sampler(y) | C_j}}
\]
\[
\Pr{Sampler(y)} = \sum_{C_j}{\Pr{Sampler(y) | C_j}\frac{q_i}{q}}
\]
Now to roll up our sleeves: what is the probability we sample our fixed vector $y$ given we have sampled a clause which it satisfies. Note that the randomness under consideration is only over the entries of $y$ which are not present in $C_j$ since our algorithm fixes those entries deterministically. Lets define the indices in $y$ which were not set by our algorithm and whose variables are positive in $y$ as $I_{pos}$ and the indices which are negated in $y$ as $I_{pos}$. Then, the probability that we sample $y$ conditioned on $C_j$ is equal to the product of these probabilities; we are computing the chance that each of these entries was set:
\[
\prod_{i \in I_{pos}}{p_i}\prod_{j \in I_{neg}}{1-p_j}
\]
Here comes the trick; observe what occurs to this above expression if we multiple both the numerator and denominator by the probability that $y$ satisfies $C_j$:
\[
\frac{\prod_{i \in I_{pos}}{p_i}\prod_{j \in I_{neg}}{1-p_j} * q_i}{q_i}
\]
The numerator is exactly the probability of sampling $y$ in the original space discussed above where a vector is drawn from the weights of $p_1 \dots p_n$:
\[
\Pr{Sampler(y)} = \sum_{C_j}{\frac{w(y)}{q_i}\frac{q_i}{q}} = \frac{N(y)w(y)}{q}
\]
\[
\Exp{\hat{p}} = \sum_{y\in U}{\frac{N(y)w(y)}{q}*\frac{q}{N(y)} = \sum_{y\in U}}w(y) = p^*
\] \\

Second, I claim that $\hat{p}$ has low variance:
\[
\Var{\hat{p}} \leq \Exp{\hat{p}^2} \leq M\Exp{\hat{p}}
\]
where $M$ is the max value which the random variable $\hat{p}$ can take on. Note that $q_i \leq p*$ since $p^*$ is the probability that any of the clauses are satisfied which is at least as large as the probability that exactly $1$ is satisfied. Since $N(y) \geq 1$, we have that:
\[
\hat{p} \leq q = \sum_{i=1}^{m}{q_i} \leq mp^* \implies M \leq mp^*
\]
\[
\Var{\hat{p}} \leq mp^{*2}
\]
From here, since we have obtained an unbiased estimator with an estimator quality that is good ($\frac{\Var{\hat{p}}}{\Exp{\hat{p}}^2} \leq m)$ we can apply the boosting theorem and obtain an $(\epsilon,\delta)$ estimator by sampling $\hat{p}$ $k = O(\frac{m log(2/\delta)}{\epsilon^2})$ times and taking the median. \qed
\end{solution}
\end{document}