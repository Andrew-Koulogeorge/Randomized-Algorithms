\usepackage{lipsum}
\documentclass[12pt]{article}
\usepackage{ifthen}
\usepackage{tikz-cd}
\usepackage{soul}
\usepackage{fullpage, prettyref}
\usepackage{fancyhdr}
\usepackage{palatino}
\usepackage[T1]{fontenc}
\usepackage{ulem}
\usepackage{graphicx}
\usepackage{ifthen}
\usepackage{algorithm}

\usepackage{a4}
\usepackage[margin=1in]{geometry} 
\usepackage[noend]{algpseudocode}
\usepackage{amsmath, amssymb, amsthm, mathtools}
\usepackage{setspace}
\usepackage{multicol}
\usepackage{array}
\usepackage{enumitem}
\usepackage{csquotes}
\usepackage{url}
\usepackage{boxedminipage}
\usepackage{wrapfig}
\usepackage{color, xcolor}
\usepackage{transparent}
\usepackage{parskip}
\usepackage{varwidth}
\usepackage{bm}
\usepackage{xspace}
\usepackage{url}
\usepackage{fullpage, prettyref}
\usepackage{boxedminipage}
\usepackage{wrapfig}
\usepackage{ifthen}
\usepackage{framed}
\usepackage{mdframed}
\usepackage[colorlinks=true,urlcolor=blue,linkcolor=blue,citecolor=violet,pdfstartview=FitH]{hyperref}
\usepackage[noabbrev,nameinlink]{cleveref}
\usepackage{enumitem}
\usepackage{tikz}
\usepackage{mathrsfs} 
\usepackage{bbm}
\usepackage[backend=biber,bibencoding=ascii,style=alphabetic,sorting=none]{biblatex}

% config
\renewcommand{\headrulewidth}{0pt}
\renewcommand{\footrulewidth}{0pt}
\pagestyle{fancy}
\fancyhead{}
\fancyfoot{}
\fancyfoot[R]{\thepage}

% Algebra
\newcommand{\unit}[1]{#1^{\times}}
\DeclareMathOperator{\Orb}{Orb}
\DeclareMathOperator{\Stab}{Stab}
\DeclareMathOperator{\Hom}{Hom}
\DeclareMathOperator{\End}{End}
\DeclareMathOperator{\Aut}{Aut}
\DeclareMathOperator{\Gal}{Gal}
\DeclareMathOperator{\rank}{rk}
\DeclareMathOperator{\img}{img}
\DeclareMathOperator{\id}{id}
\DeclareMathOperator{\ch}{ch}
\DeclareMathOperator{\tr}{tr}
\DeclareMathOperator{\codim}{codim}
\DeclareMathOperator{\ann}{ann}
\DeclareMathOperator{\ev}{ev}
\DeclareMathOperator{\proj}{proj}
\DeclareMathOperator{\ang}{ang}
\DeclareMathOperator{\GL}{GL}
\DeclareMathOperator{\Ob}{Ob}
\newcommand{\category}[1]{\mathsf{\underline{#1}}}
\newcommand{\acts}{\circlearrowleft}
\newcommand{\tleq}{\unlhd}

% Boolean logic/algorithms
\newcommand{\cor}{\textsc{ or }}
\newcommand{\cand}{\textsc{ and }}
\newcommand{\xor}{\oplus}
\newcommand{\true}{\textsc{True}}
\newcommand{\false}{\textsc{False}}
\newcommand{\nil}{\textsc{Nil}}
\newcommand{\alg}[1]{\textsc{#1}}
\DeclareMathOperator{\cmod}{mod}
\newcommand{\eq}{\leftarrow}
\newcommand{\chapterref}[2]{[\S #1, #2]}

% math formatting
\def\bar{\overline}
\newcommand{\script}[1]{\mathcal{#1}}
\newcommand{\fancyscript}[1]{\mathscr{#1}}


% number sets
\newcommand{\N}{\mathbb{N}}
\newcommand{\Q}{\mathbb{Q}}
\newcommand{\R}{\mathbb{R}}
\newcommand{\Z}{\mathbb{Z}}
\newcommand{\C}{\mathbb{C}}
\newcommand{\F}{\mathbb{F}}

%Paired Delims
\def\set#1{\left\{ #1 \right\}}
\def\angle#1{\left\langle #1 \right\rangle}
\def\sqbrackets#1{\left[ #1 \right]}
\def\brackets#1{\left( #1 \right)}
\DeclarePairedDelimiter\ceil{\lceil}{\rceil}
\DeclarePairedDelimiter\floor{\lfloor}{\rfloor}
\newcommand{\abs}[1]{\left\lvert #1 \right\rvert}

%problem environment
\newenvironment{problem}[2][Problem]{\begin{trivlist}
\item[\hskip \labelsep {\bfseries #1}\hskip \labelsep {\bfseries #2.}]}{\end{trivlist}
\vspace{1em}
}

%problems list environment
\newlist{subprob}{enumerate}{1}
\setlist[subprob]{label={\bfseries\itshape\alph*.}}

%solution environment
\newenvironment{solution}[1][Solution]{
    \setcounter{nlem}{0}
    \vspace{-2ex}
  \begin{proof}[#1]
}{
  \phantom\qedhere\end{proof}
}


% colored boxes
\newenvironment{answerbox}{
      \begin{mdframed}[backgroundcolor=green!30, innertopmargin=12px, topline=false,rightline=false,leftline=false,bottomline=false]}{
  \end{mdframed}%
}

\newenvironment{thmbox}{
        \vskip2ex
          \begin{mdframed}[backgroundcolor=blue!10, topline=false,rightline=false,leftline=false,bottomline=false]}{
  \end{mdframed}
}

\newenvironment{algbox}[1]{
        \vskip2ex
            \begin{mdframed}[backgroundcolor=yellow!10, topline=false,rightline=false,leftline=false,bottomline=false, userdefinedwidth=.7\linewidth, align=center]
              \vspace{1ex}
              \hrulefill
              \vspace{-1.5ex}
  
              {\bf #1}
  
              \vspace{-2.5ex}
              \hrulefill
              }{
            \end{mdframed}
}

\newenvironment{algboxw}[2]{
        \vskip2ex
            \begin{mdframed}[backgroundcolor=yellow!10, topline=false,rightline=false,leftline=false,bottomline=false, userdefinedwidth=#2\linewidth, align=center]
              \vspace{1ex}
              \hrulefill
              \vspace{-1.5ex}
  
              {\bf #1}
  
              \vspace{-2.5ex}
              \hrulefill
              }{
            \end{mdframed}
}

\newenvironment{algboxnt}{
        \vskip2ex
            \begin{mdframed}[backgroundcolor=yellow!10, topline=false,rightline=false,leftline=false,bottomline=false, userdefinedwidth=.7\linewidth, align=center]\vspace{1.5ex}}{
            \end{mdframed}
}

% theorem envs
\theoremstyle{definition}
\newtheorem*{lem}{Lemma}
\newtheorem*{claim}{Claim}
\newtheorem*{thm}{Theorem}
\newtheorem*{defn}{Definition}
\newtheorem*{coro}{Corollary}
\newtheorem*{rmk}{Remark}
\newtheorem*{prop}{Proposition}
\newtheorem*{example}{Example}
\newtheorem*{probstatement}{Problem}
\newtheorem{nlem}{Lemma}
\newtheorem{ndefn}[nlem]{Definition}
\newtheorem{nthm}[nlem]{Theorem}
\newtheorem{ncor}[nlem]{Corollary}
\newtheorem{nprop}[nlem]{Proposition}
\newtheorem{nrmk}[nlem]{Remark}


% Probability
\newcommand{\Prob}[1]{\mathbb{P}\left[#1\right]}
\newcommand{\Exp}[1]{\mathbf{E}\left[#1\right]}
\newcommand{\Var}[1]{\mathbf{Var}\left[#1\right]}
\newcommand{\Cov}[1]{\mathbf{Cov}[#1]}
\renewcommand{\Pr}{\Prob}
\DeclareMathOperator{\Bin}{Bin}
\DeclareMathOperator{\Pois}{Pois}
\DeclareMathOperator{\Geom}{Geom}
\DeclareMathOperator{\Normal}{\mathcal{N}}


% Misc commands
\newcommand{\sgn}{\mathrm{sgn}}
\newcommand{\poly}{\mathrm{poly}}
\newcommand{\polylog}{\mathrm{polylog}}
\newcommand{\contradiction}{\hfill$\Rightarrow\!\Leftarrow$}
\newcommand{\writer}{Andrew Koulogeorge, Eric Richardson}
\newcommand{\email}{andrew.j.koulogeorge.24@dartmouth.edu}
\newcommand{\TODO}[1]{\textcolor{red}{\bf [TODO: #1] }}
\newcommand{\note}[1]{\textcolor{blue}{\bf [Note: #1] }}
\newcommand{\alert}[1]{\textcolor{red}{\bf [#1]}}
\renewcommand{\epsilon}{\varepsilon}
\renewcommand{\phi}{\varphi}
\newcommand{\lnorm}[2]{\left\lVert#1\right\rVert_{#2}}
\DeclareMathOperator*{\argmin}{argmin}
\DeclareMathOperator*{\argmax}{argmax}
% PSet Titles
\newcommand{\normaltitle}[3]{
    \begin{center}
        {\LARGE #2}
        \vskip-1ex
        {\large \writer}
        \vskip-1ex
        {\large Date: \today}
        \vskip-1ex
        {\large Course: #1, ~#3}
        \vskip0.1ex
        \vspace{2ex}
        \hrule height 2pt
        \vspace{2ex}
    \end{center}
}

\newcommand{\fancytitle}[3]{
	{
            \rule{\textwidth}{1pt}
		\begin{center}
			{
                    \setlength{\parskip}{0mm} \setlength{\parindent}{0mm}
				{\textbf{#2} \hfill \writer}\\
				#1:~#3 \hfill Date: \today
                }
		\end{center}
            \vskip-2ex
            \rule{\textwidth}{1pt}
	}
 }
\everymath{\displaystyle}

\begin{document}

\fancytitle{COSC 34}{Problem Set 3}{Randomized Algorithms}

Collaboration Statement: Andrew Koulogeorge and Eric Richardson 

\begin{problem}{5: Taking Care of False Positives}
\end{problem}

\begin{solution}

\begin{answerbox}
\begin{algorithmic}[1]
\Procedure{Sampler}{$K, \epsilon, \delta$}
\State Sample from given importance sampling algorithm $N = \frac{S\log{2n/\delta}}{k\epsilon^2}$ times and keep an hash map of frequencies for how many times we have seen each element. 
\State Include an element in $H$ if we sampled it at least $(1-\epsilon)\frac{NK}{S}$ times
\State return H
\EndProcedure
\end{algorithmic}
\end{answerbox}
Let us define two random variables $X$ and $Y$ where $X$ equals the number of times we sample a fixed element $x \in U~ s.t~ score(x) \geq k$ in $N$ samples and $Y$ equals the number of times we sample a fixed element $y \in U~ s.t ~score(y) < (1-\epsilon)k$. For shorthand, let us define $T = \frac{NK}{S}$. This gives us a threshold value of $T(1-\frac{\epsilon}{2})$. We wish to bound the probability that $X$ is smaller than our threshold and the probability that $Y$ is larger than the threshold. If $X$ is smaller than the threshold, that means that there existed a heavy element which we didn't include in H (bad recall) and if $Y$ is larger than the threshold that means there existed a light element which we did include (bad precision). If we can bound the probability for a single heavy element $x$ not occurring enough and bound the probability for a single light element occurring too often, then we can apply union bound and bound the overall probability of failure. \\

Lets begin by computing the expected value for $X$ and $Y$. First, observe that $X$ and $Y$ can be expressed as sum of independent Bernoulli Random variables where $X_i = 1$ if the ith sample drawn from the importance sampler is $x$ and $0$ otherwise. A symmetric defining is applied to $Y$. Now observe that since we are using a given importance sampler, the probability that $x$ or $y$ is sampled using $A$ is $\frac{score(x)}{S}$ and $\frac{score(y)}{S}$ respectively. By applying the law of total expectation we can see that:
\[
\Exp{X} = \sum_{i=1}^{\frac{TS}{K}}{\frac{score(x)}{S}} = \frac{TS}{K}\frac{score(x)}{S} = \frac{T * score(x)}{K} \geq T
\]
\[
\implies \Exp{X} \geq T
\]
\[
\Exp{Y} = \sum_{i=1}^{\frac{TS}{K}}{\frac{score(y)}{S}} = \frac{TS}{K}\frac{score(y)}{S} = \frac{T * score(y)}{K} < T (1-\epsilon)
\]
\[
\implies \Exp{Y} < T(1-\epsilon)
\]
First, lets bound the probability of $X$ being less than the threshold. Let $q = \frac{\epsilon}{2} \in (0,1)$
\[
\Pr{X \leq (1-q)\Exp{X}} \leq \exp{O(-q^2\Exp{X})}
\]
Since $\Exp{X} \geq T$:
\[
\Pr{X \leq (1-q)T} \leq \Pr{X \leq (1-q)\Exp{X}} \leq \exp{O(-q^2\Exp{X})}
\]
Plugging in for $q = \frac{\epsilon}{2}$:
\[
\Pr{X \leq (1-\frac{\epsilon}{2})T} \leq \exp{O(-\epsilon^2\Exp{X})} \leq \exp{O(-\epsilon^2T)}
\]
Thus, we have bounded the probability that a given element $x \in U$ where $score(x) \geq k$ is sampled less than our threshold. We can now take the union bound over \textbf{any} element $x$ failing in this way to obtain a bound on the probability of this mistake occurring for any element in $U$:
\[
\Pr{\textit{Probability that any element falls below the threshold}} \leq n\exp{O(-\epsilon^2T)}
\]
Note that this upper bound is pretty course since we are even considering elements that are not heavy, but it does the trick.\\

Now lets apply the same procedure for the false positive case $Y$. Let $w = \frac{\epsilon}{2}$. Applying Chernoff on Y:
\[
\Pr{Y \geq (1+w)\Exp{Y}} \leq \exp{O(-w^2\Exp{X=Y})}
\]
Since $\Exp{Y} < T(1-\epsilon)$:
\[
\Pr{Y \geq (1-\epsilon)(1+w)T} \leq \Pr{Y \geq (1+w)\Exp{Y}} \leq \exp{O(-w^2\Exp{Y})}
\]
Now lets analyze how our chosen value for $w = \frac{\epsilon}{2}$ compares to the threshold value of $(1-\frac{\epsilon}{2})T$. When we plug in $w$ we get that:
\[
(1-\epsilon)(1+w)T = (1-\epsilon)(1+\frac{\epsilon}{2})T = (1 - \frac{\epsilon}{2} - \frac{\epsilon^2}{4})T < (1 - \frac{\epsilon}{2})T
\]
Since $(1-\epsilon)(1+w)T < (1 - \frac{\epsilon}{2})T$:
\[
\Pr{Y \geq (1 - \frac{\epsilon}{2})T} \leq \Pr{Y \geq (1-\epsilon)(1+w)T} \leq \exp{O(-\epsilon^2\Exp{Y})} \leq \exp{O(-\epsilon^2T)}
\]
Lastly, note that we got rid of the expectation of $Y$ in the exponent but we didn't use the same argument as the analysis with $X$. Recall from the fine print of DeepC's notes that the bounds from Chernoff still apply when $\Exp{Y}$ is upper bounded. Therefore, we substituted in $\Exp{Y} < T(1-\epsilon)$. Now, we apply union bound like we did for the false negative case to bound the bad event of an example being a false negative across all examples in $U$
\[
\Pr{\textit{Probability that any element falls above the threshold}} \leq n\exp{O(-\epsilon^2T)} 
\]
With this analysis we have shown that the probability that our algorithm fails, that is, the probability that there exists a heavy element $x \in U$ which is not in $H$ \textbf{OR} that there exists a light element $y \in U$ which is in H is bounded above by the following expression. By including either of these cases as a failure, we are able to ensure both high precision and recall with high probability. We wish to solve for the number of samples $N$ our algorithm must take from $U$ in order to ensure that the probability of failure is at most $\delta$ so lets rewrite $T$ in its form of $\frac{NK}{S}$:
\[
\Pr{\textit{fail}} \leq 2n\exp{O(-\epsilon^2\frac{NK}{S})} = \delta
\]
\[
\implies N > \frac{S}{K} \frac{1}{\epsilon^2} \log(2n/\delta) = O(\frac{S}{K} \frac{1}{\epsilon^2} \log(n/\delta)) \qed
\]



\end{solution}
\end{document}