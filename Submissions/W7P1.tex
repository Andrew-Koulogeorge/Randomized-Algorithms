\documentclass[12pt]{article}
\usepackage{mathtools}
\DeclarePairedDelimiter\ceil{\lceil}{\rceil}
\DeclarePairedDelimiter\floor{\lfloor}{\rfloor}
\usepackage{lipsum}

\input{macros}
\everymath{\displaystyle}

\begin{document}

\fancytitle{COSC 34}{Problem Set 7}{Randomized Algorithms}

Collaboration Statement: Andrew Koulogeorge and Eric Richardson 

\begin{problem}{1: Heavy Hitters}
\end{problem}

\begin{solution} \ 

\begin{answerbox}
\begin{algorithmic}
\Procedure{HeavyHitters}{$\delta, \epsilon, \phi$}
\State $m \leq \hat{m} \leq 10m~$ (by assumption)
\State Initialize $CountMin(\epsilon, \frac{\delta}{\hat{m}})~$ (Data structure from lecture)
\State Initialize $MinHeap~$ (Keep track of HH candidates)
\State $\hat{M}=0~$ (Keep track of elements in stream so far)
\For{$x\in stream$}
    \State $\hat{M} += 1$
    \State Increment all counters in $CountMin$
    \State $\hat{f_x} = Count(x)$ (Get estimate of frequency of $x$ from $CountMin$)
      \State Pop from $MinHeap$ if $|MinHeap| > \frac{1}{\phi - \epsilon}$
    \If{$\hat{f_x} \geq M\phi$}
        \State Push $(\hat{f_x},x)$ onto $MinHeap$ 
    \EndIf
\EndFor
\State Assert $\hat{M} = m$ after the last element in $stream$ goes by
\State $R = \{\}$
\For{$x\in MinHeap$}
    \State $\hat{f_x} = Count(x)$
    \If{$\hat{f_x} \geq \hat{M}\phi$}
        \State Add to $R$
    \EndIf
\EndFor
\State Return $R$

\EndProcedure
\end{algorithmic}
\end{answerbox}

\textbf{Probability Guarantee} Consider a bad element $y$ such that $f_y \leq (\phi - \epsilon)m$ but $\hat{f_y} \geq m\phi$. Recall the theorem from lecture that states for a $MinCount$ data structure with $\ln{\frac{m}{\delta}}$ hash functions we have that:
\[
\Pr{f_y + m\epsilon < \hat{f_y}} < \frac{\delta}{m}
\]
\[
\implies \Pr{m\phi \leq \hat{f_y}} \leq \Pr{f_y + m\epsilon < \hat{f_y}} < \frac{\delta}{m}
\]
Where the second implication follows from 
$f_y \leq (\phi - \epsilon)m \implies f_y + m\epsilon \leq m\phi$. In short, the probability that a bad element $y$ fools our $MinCount$ into thinking it is a heavy hitter is at most $\frac{\delta}{m}$ for a fixed $y$. Applying union bound over all $m$ elements in the stream, we get the probability of a any bad element fooling us to be at most $\delta$. \qed

For the rest of the analysis assume we are in this nice case that no bad elements $y$ exist. 
Now we need to show that in this case that occurs at least $1-\delta$ of the time, we meet the various requirements.


\textbf{Space Requirement + Correctness:} We keep a $CountMin$ data structure with $O(\frac{1}{\epsilon}\log{\frac{m}{\delta}})$ words since $m \leq \hat{m} \leq 10m$. We also keep a $MinHeap$ to store heavy hitter candidates. To ensure that our algorithm is correct, we must ensure that all heavy hitters in the heap after the stream passes. Assume the size of the heap is some constant $c$ and we will derive a necessary lower bound. 

To begin, fix a heavy hitter $z$ and consider the last time $z$ appears in the stream. Since $CountMin$ always overestimates the true frequency of an element, $z$ is a heavy hitter and $m \geq M$ at all points in the stream:
\[
\hat{f_z} \geq f_z \geq m\phi \geq M\phi
\]
This implies for every heavy hitter $z$, it will be in the heap after the last time it passes in the stream. Now we need to show that its impossible for this heavy hitter $z$ to be evicted after its last instance has passed in the stream. Suppose for contradiction that $z$ was popped from $MinHeap$ after it passed the stream for the last time. This implies that it has the smallest frequency in $MinHeap$:
\[
\forall~j\in MinHeap:~\hat{f_j} \geq \hat{f_z} \geq f_z \geq m\phi \geq M\phi
\]
This is, all elements $j$ currently in the heap have the property that $\hat{f_j} \geq m\phi$. Since we are reasoning in the case where $\nexists ~y~s.t~\hat{f_y} \geq m\phi$ and $f_y \leq m(\phi-\epsilon) \implies f_j >m(\phi-\epsilon)~\forall~j\in MinHeap$. This implies the total number of elements we have seen in the stream so far $\geq cf_j > cm(\phi-\epsilon)$. If $c \geq \frac{1}{\phi -\epsilon} \implies $ elements seen in the stream so far $>m$, a contradiction. Thus, as long as the size of the heap is at least as large as $\frac{1}{\phi -\epsilon}$, $MinHeap$ will contain all the heavy hitters with probability at least $1-\delta$. If $\epsilon < \frac{\phi}{2} \implies 2\epsilon < \phi \implies \epsilon < \phi - \epsilon \implies \frac{1}{\epsilon} > \frac{1}{\phi -\epsilon} \implies$ heap size is $O(\frac{1}{\epsilon})$ \qed

Thus, the total space of the algorithm is dominated by the $MinCount$ table: $O(\frac{1}{\epsilon}\log{\frac{m}{\delta}})$

\textbf{During Stream Time Requirement:} When an element $x$ passed by in the stream, we update all the counters in the $CountMin$ data structure. There are $\log{\frac{m}{\delta}}$ counters to update and we assume we can compute each hash function in constant time. we then compute the current frequency estimate of $x$ which takes the $min$ across all $\log{\frac{m}{\delta}}$ counters. If $x$ is estimated to be a heavy hitter so far, we push it onto the heap which takes $O(\log{\frac{1}{\epsilon}})$ time since we argued that the heap size is $O(\frac{1}{\epsilon})$. Removing candidates when the heap becomes full also takes $O(\log{\frac{1}{\epsilon}})$. Since we only evict from the heap when it fills up, we do at most $2$ heap operations for each element in the stream. Thus, the total time taken for a single element in the stream $=O(\log{\frac{m}{\delta}}) + O(\log{\frac{1}{\epsilon}}) = O(\log{\frac{m}{\delta}})$. (Assuming $\frac{m}{\delta} > \frac{1}{\epsilon}$)

\textbf{Post Stream Time Requirement:} After all the elements in the stream pass, we loop over each element in the heap, compute the estimate for its frequency, and place it in $R$ if our estimate tells us its a heavy hitter. Since our heap is of size $O(\frac{1}{\epsilon})$ and each query to $MinCount$ takes $O(\log{\frac{m}{\delta}})$ time $\implies $Post Stream Time Requirement = $O(\frac{1}{\epsilon}\log{\frac{m}{\delta}})$

\end{solution}


\end{document}